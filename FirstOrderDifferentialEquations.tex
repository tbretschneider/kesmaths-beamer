\documentclass[8pt]{beamer}

 \usepackage[utf8]{inputenc}                                                     
 \usetheme[block=fill,progressbar=foot,background=light]{metropolis}                                                               
%  \usecolortheme{crane}                                                       
  %\useinnertheme{circles}                                                         
  \usepackage[english]{babel}                                                     
  \usepackage{csquotes}                                                           
  \usepackage[T1]{fontenc}                                                        
  \usepackage{booktabs}                                                           
  \usepackage{amsmath}   
  \usepackage{tikz}
  \usepackage{amssymb}
  \usepackage{amsfonts}
  \usepackage{mathrsfs}   
  \usepackage{graphicx}
  \usepackage{varioref}
  \usepackage{probsoln}
  \usepackage{pgfplots}
\pgfplotsset{compat=newest}
  \usepackage[style=authoryear,backref=true]{biblatex}
 \usepackage[]{hyperref} 
  \graphicspath{{Graphics/}}
  \usepackage{multirow,array}
  \addbibresource{../Everything.bib}
  \usepackage{colortbl}
  \definecolor{aa}{RGB}{255, 124, 0}
  \definecolor{cc}{RGB}{230, 230, 230}    
  %\setbeamercolor{palette tertiary}{fg=aa,bg=cc}
  %\setbeamercolor{structure}{fg=cc}
  %\setbeamercolor{alerted text}{fg=red}
  
  %Information to be included in the title page:
  \title[Discrete]{AL FM Discrete}
  
  \subtitle{First Order Differential Equations}
  
  \author[]{T. Bretschneider}
  
  \date[\today]{\today}

\usepackage{comment}
\usepackage{varwidth}

\newcommand{\mat}[4]{\left(\begin{array}{cc} #1 & #2 \\ #3 & #4 \\ \end{array}\right)}
\newcommand{\Q}{\mathbb{Q}}
\newcommand{\R}{\mathbb{R}}
\newcommand{\Z}{\mathbb{Z}}
\newcommand{\sol}[2][+]{
	\tikz[baseline]{\node[color=aa,fill=cc,rectangle,draw,anchor=base] {  {\onslide<#1->{#2}}  };}
}

\usetikzlibrary{positioning}
\usetikzlibrary{tikzmark}



\def\height{0.8cm}
\def\width{1.2cm}

		\newcommand{\keynode}[6]{\node[minimum height=\height,minimum width=\width,draw,rectangle,color=aa,fill=cc] (#3) at (#1,#2) {};
	\node[rectangle,minimum height=\height/2,minimum width=\width,above,color=aa,fill=cc] at (#3) {#3};
	\node[draw,rectangle,minimum height=\height/2,minimum width=\width/3,below,color=aa,fill=cc,inner sep =0cm] at (#3) {\footnotesize#4};
	\node[draw,rectangle,minimum height=\height/2,minimum width=\width/3,below,xshift=\height/2,color=aa,fill=cc,inner sep=0cm] at (#3) {\footnotesize#5};
	\node[draw,rectangle,minimum height=\height/2,minimum width=\width/3,below,xshift=-\height/2,color=aa,fill=cc,inner sep=0cm] at (#3) {\footnotesize#6}; }



\begin{document}

\setlength{\abovedisplayskip}{0pt}
\setlength{\belowdisplayskip}{0pt}
\setlength{\abovedisplayshortskip}{0pt}
\setlength{\belowdisplayshortskip}{0pt}


\frame{\titlepage}


\begin{frame}[shrink]{Differential Equations}
	\begin{definition}
		\textbf{Differential Equations} are equations which include the derivatives of the variable.
	\end{definition}
	\alert<1>{Typically we will start with an equation which involves derivatives and aim to get the equation which just relates the two variables to each other.}
	
		\begin{columns}
			\begin{column}{0.6\linewidth}
				\begin{example}
					\begin{itemize}
						\item The rate of temperature loss is proportional to the current temperature.
						\item The rate of population change is proportional to $P\left( 1-\frac{P}{M} \right) $ where $P$ is the current population and $M$ is the limiting size of the population. (the Verhulst-Pearl Model)
						\item Suppose $x$ is GDP (Gross Domestic Product). Rate of change of GDP is proportional to current GDP.
					\end{itemize}
				\end{example}
			\end{column}
			\begin{column}{0.4\linewidth}
				\begin{itemize}
					\item \[
					\frac{dT}{dt}=-kT
					.\] 
					\item \[
					\frac{dP}{dt}=-kP\left( 1-\frac{P}{M} \right)               
					.\] 
					\item \[                    
					\frac{dx}{dt}=kx
					.\] 
				\end{itemize}
			\end{column}	
		\end{columns}
		

	They are used a lot in physics and engineering, including modelling radioactive decay, mixing fluids, cooling materials, shock absorbance and bodies falling under gravity against resistance.
	\begin{definition}
		A \textbf{first order differential equation} mean the equation contain the first derivative, $\frac{dy}{dx}$, but not the second derivative nor beyond. 
	\end{definition}
\end{frame}

\begin{frame}{A-level Revision: Separation of Variables}
\begin{align*}
	\frac{dy}{dx}&=\tikzmarknode{a}{f(x)g(y)} \\
	\frac{1}{g(y)}\frac{dy}{dx} &= \tikzmarknode{b}{f(x)}\\
	\frac{1}{g(y)}dy &= f(x)\tikzmarknode{c}{dx} \\
	\tikzmarknode{d}{\int} \frac{1}{g(y)}dy &= \int f(x) {dx} 
.\end{align*}	

\begin{alertblock}{Exceptions}
	However, we would not be able to use this trick on the differential equation:
	\[
		x^3\frac{dy}{dx}+3x^2y=\sin(x)
	.\] 
	
\end{alertblock}

\begin{center}
\begin{tikzpicture}[overlay,remember picture,scale=1]
	\draw[color=aa,thick,<-] (a.north) --++ (-0.5,0.5) node[above, align=left] {$x$ and $y$ are said to be \emph{separable} because there is \\ just one product of a function of $x$ and a function of $y$.};
	\draw[color=aa,thick,<-] (b.east) --++ (1,-0.5) node (e) {};
		\draw[color=aa, thick,<-] (c.east) -- (e) node[right,align=left] {Divide through by $g(y)$ and \\ then multiply by  $dx$.}; 
		\draw[color=aa,thick,<-] (d.west) --++ (-1.5,0.5) node[above,align=left] {Put an integral symbol \\ on the front!};
\end{tikzpicture}
\end{center}


\end{frame}


\begin{frame}[shrink=8]{General Solution}
	\begin{definition}
		If we are just given a differential equation then there is a family of equations that could be the final answer which we call the \textbf{general solution} .
	\end{definition}

	\begin{problem}
		Find general solutions to $\frac{dy}{dx}=2$.
	\end{problem}

	\sol{$y=2x+C$}, Why is it called the general solution?

	\sol{There is a 'family' of solutions as the constant of integration varies.}

\begin{problem}
	Find general solutions to $\frac{dy}{dx}=-\frac{x}{y}$.
\end{problem}


	\begin{columns}[T]
		\begin{column}{.4\linewidth}
			\begin{solution}<3->
	\begin{align*}
		\int y dy &= \int -x dx \\
		\frac{1}{2}y^2 &= -\frac{1}{2}x^2+C \\
		x^2+y^2&= 2C \\
		\text{If we let } r^2&= 2C \\
		x^2+y^2&=r^2
	.\end{align*}
	\end{solution}
\end{column}
\begin{column}{.6\linewidth}
	\onslide<4->{
	So the 'family of circles' satisfies this D.E.
				\begin{tikzpicture}
					\begin{axis}[mlineplot,width=4cm,height=4cm,
						xmin= -1, xmax= 1,
						ymin= -1, ymax = 1,
						axis lines = middle,
					]
					\addplot[domain=-180:180, samples=100]({cos(x)},{sin(x)});
					\addplot[domain=-180:180,samples=100]({0.5*cos(x)},{0.5*sin(x)});
					\addplot[domain=-180:180,samples=100]({0.25*cos(x)},{0.25*sin(x)});

					\end{axis}
				\end{tikzpicture}}
		\end{column}
	\end{columns}


\end{frame}

\begin{frame}[shrink]{Test Your Understanding}
	\begin{columns}[T]
		\begin{column}{0.5\textwidth}
	\begin{problem}
	Find general solution to $\frac{dy}{dx}=xy+x$.
	\end{problem}
	\begin{solution}<2->
		\begin{align*}
			\frac{dy}{dx}&=x(y+1) \\
		\int \frac{1}{y+1} dy &= \int x dx \\
		\ln (y+1) &= \frac{1}{2}x^2+C \\
		y+1&=e^{\frac{1}{2}x^2+C} \\
		   y &= Ae^{\frac{1}{2}x^2}-1 
		.\end{align*}
	\end{solution}
\end{column}
\begin{column}{.5\textwidth}
	\begin{problem}
		Find general solutions to $\frac{dy}{dx}=-\frac{y}{x}.$
	\end{problem}
	\begin{solution}<3->
		\begin{align*}
			\int \frac{1}{y}dy&= \int -\frac{1}{x}dx \\
			\ln |y| &= -\ln x +C \\
			\ln x + \ln |y| &= C\\
			\ln xy &= C \\
			|xy| &= e^{C} = A \\
			y &= \pm \frac{A}{x}
		.\end{align*}
		A family of rectangular hyperbola. 
	\end{solution}
\end{column}
\end{columns}
\end{frame}

\begin{frame}[shrink=2]{Particular Solutions}
	\begin{definition}
		In an A-level question we usually only end up with one solution because...

		They give us the details of one point (sometimes called \emph{initial condition}) which must be satisfied by the solution. We can the substitute this into the general solution and this gives us a \textbf{particular solution} .
		

	\end{definition}

	\begin{problem}
		Find the particular solution for $\frac{dx}{dt}=\sqrt{x} $ which satisfy the initial conditions $x=9$ when $t=0.$

	\end{problem}

	\begin{solution}<2->
		\begin{align*}
			\int \frac{1}{\sqrt{x} } dx &= \int 1 dt \\
			2\sqrt{x} &= t+c \\
			2(3) &= 0+c \implies c=6 \\
			2\sqrt{x} &= t+6 \\
			x&= \left( \frac{t+6}{2} \right)^2 = \frac{(t+6)^2}{4}  
		.\end{align*}
	\end{solution}

	\alert<2>{Some people (including myself) like to sub in as soon as possible before rearranging.}
\end{frame}

\begin{frame}{Reversing the Product Rule}
	\begin{problem}
		Find general solutions of the equation $x^3 \frac{dy}{dx}+ 3x^2y=\sin x$.
	\end{problem}

	We can't \sol{separate} the variables. But do you notice anything about the LHS?

	\sol{It's $\frac{d}{dx}\left( x^3y \right) $.}

	Quickfire Questions:

	\begin{columns}
		\begin{column}{.5\textwidth}
			$\frac{d}{dx}\left( x^2y \right) =$\sol{$2xy+x^2 \frac{dy}{dx}$.}

			$\frac{d}{dx}\left( y\sin(x) \right) =$ \sol{$\cos(x)y + \sin(x) \frac{dy}{dx}$.}
	
\end{column}
\begin{column}{.5\textwidth}
	$x^{4} \frac{dy}{dx}+4 x^3y = $ \sol{$\frac{d}{dx}\left( x^{4}y \right) $.}

		$e^x \frac{dy}{dx}+ e^x y=$ \sol{$e^x y$.}
	
		$(\ln x) \frac{dy}{dx}+ \frac{y}{x}=$ \sol{$ \ln(x)y$.}
\end{column}
		
	\end{columns}
	  
	\alert<5->{So it appears whatever term ends up on front of the $\frac{dy}{dx}$ will be on front of the $y$ in the integral.}
\end{frame}

\begin{frame}[shrink]{Differential Equations by Reversing the Product Rule}
	\begin{problem}
		Find general solutions of the equation $x^3 \frac{dy}{dx} + 3x^2 y = \sin x$.
	\end{problem}

	\begin{solution}<2->
		\begin{align*}
			\frac{d}{dx}(x^3y) &= \sin x \\
			x^3y &= \int \sin x dx = -\cos x +C \\
			y &= \frac{C-\cos x}{x^3}
		.\end{align*}
	\end{solution}

\begin{columns}
\begin{column}{.5\linewidth}
\begin{problem}
	Find general solutions of the equation $\frac{1}{x} \frac{dy}{dx}-\frac{1}{x^2}y=e^{x}$.
\end{problem}
\begin{solution}<3->
	\begin{align*}
		\frac{d}{dx}\left( \frac{y}{x} \right) &= e^{x} \\
		\frac{y}{x} &= e^{x}+c \\
		y &= x(e^{x}+c)
	.\end{align*}
\end{solution}
\end{column}
\begin{column}{.5\linewidth}
\begin{problem}
	Find general solution of the equation $4xy \frac{dy}{dx}+2y^2=x^2$.
\end{problem}
\begin{solution}<4->
	\begin{align*}
		\frac{d}{dx}(2xy^2)&= x^2 \\
		2xy^2 &= \frac{1}{3}x^3+C \\
		      &\ldots
	.\end{align*}
\end{solution}
\end{column}
\end{columns}



\end{frame}

\begin{frame}{Solving $\frac{dy}{dx}+ Py=Q$}
	Even if the coefficient of $y$ isn't the derivative by the coefficient of $\frac{dy}{dx}$ there is a cunning trick which means that this becomes the case and we can reverse Th product rule.

	\begin{definition}
		We can multiply through by the \textbf{integrating factor}, $ e^{\int P dx}$. This then produces an equation where we can use the previous  reverse product rule trick. 
	\end{definition}
	
\begin{problem}
	Find the general solution of $\frac{dy}{dx}-4y=e^{x}$.
\end{problem}
\begin{columns}[T]
\begin{column}{.5\textwidth}
\sol{Integrating Factor} $=$ \sol{ $e^{\int-4dx}=e^{-4x}$}

Then multiplying through the original D.E. by the integrating factor:

\sol{$e^{-4x} \frac{dy}{dx}-4e^{-4x}y=e^{-3x}$.}

Then we can solve the usual way:
\end{column}
\begin{column}{.5\textwidth}
\begin{solution}<4->
	\begin{align*}
		\frac{d}{dx}\left( ye^{-4x} \right) &= e^{-3x} \\
		ye^{-4x}&= -\frac{1}{3}e^{-3x}+C \\
		y&=-\frac{1}{3}e^{x}+Ce^{4x}
	.\end{align*}
\end{solution}
\end{column}
\end{columns}
\end{frame}

\begin{frame}[shrink=10]{Proof that the integrating factor method works}
	\begin{problem}
		Solve the general equation $\frac{dy}{dx}+Py=Q$ where $P,Q$ are function of  $x$.
	\end{problem}

	Suppose $f(x)$ is the I.F. As usual we'd multiply by it:
	\[
		\text{\sol{$f(x) \frac{dy}{dx} + f(x) P y = f(x)Q$}}
	.\] 
	If we can use the reverse trick on the LHS, then it would be of the form:
\[
	\text{\sol{$f(x) \frac{dy}{dx}+f'(x)y$}}
.\] 
Thus comparing the coefficients of the two LHSs:
\[
	\text{\sol{$ f'(x)=f(x)P$}}
.\] 
Dividing by $f(x)$ and integrating:

\begin{solution}<4->
	\begin{align*}
		\int \frac{f'(x)}{f(x)} dx &= \int P dx \\
		\ln f(x) &= \int P dx \\
		f(x) &= e^{\int P dx}
	.\end{align*}
\end{solution}


\end{frame}


\begin{frame}[shrink]{When the Coefficient of $\frac{dy}{dx}$ isn't 1}
	\begin{problem}
		Find the general solution of $\cos x \frac{dy}{dx}+ 2y\sin x = \cos^4 x.$
	\end{problem}

	What shall we do first so that we have an equation like before? \sol{Divide by $\cos x$}

	\begin{columns}[T]
	\begin{column}{.6\linewidth}
	\begin{solution}<2->
		\[
		\frac{dy}{dx}+2y \tan x = \cos^3 x
		.\] 
	\end{solution}

	\begin{solution}<3->
		\[
			\text{I.F.}= e^{\int 2 \tan x dx}= e ^{2 \ln \sec x}=\sec^2x
		.\] 
	\end{solution}
	\begin{solution}<4->
		\[
		\sec^2 x \frac{dy}{dx} + 2y \sec^2x \tan x = \cos x 
		.\] 
		\[
			\frac{d}{dx}(y \sec^2 x)=\cos x
		.\] 
	\end{solution}
	\begin{solution}<5->
		\[
		y \sec^2x =\sin x +c
		.\] 
		\[
			y= \cos^2x(\sin x +c)
		.\] 
	\end{solution}
	\end{column}
	\begin{column}{.4\linewidth}
		\begin{block}{Step 1}
		Divide by anything on front of $\frac{dy}{dx}$.
	\end{block}
	\begin{block}{Step 2}
		Determine the integrating factor.
	\end{block}
	\begin{block}{Step 3}
		Multiply through by I.F. and use product rule backwards.
		
	\end{block}
	\begin{block}{Step 4}
		Integrate and simplify.
	\end{block}
	\end{column}
	\end{columns}


\end{frame}

\begin{frame}{Past Paper Question}
	\begin{problem}
		Find the general solution of the differential equation
		\[
		x \frac{dy}{dx} + 5y = \frac{\ln x}{x}, \quad x > 0
		.\] 
	\end{problem}

	\begin{solution}<2->
		\begin{align*}
			\frac{dy}{dx} + \frac{5}{x} y &= \frac{\ln x}{x^2} \\
			x^5 \frac{dy}{dx}+ 5x^4 y &= \ln (x) x^3 \\
			x^5y &=  \int \ln (x) x^3 \ dx \\
			x^5y &= \frac{1}{4}x^4 \ln x - \frac{1}{16x^4}+c\\
			y &= \frac{1}{4x}(\ln x - \frac{1}{4}+\frac{c}{5})
		.\end{align*}
	\end{solution}
\end{frame}


\begin{frame}[shrink]{Making a Substitution}
	Sometimes making a substitution will turn a more complex differential equation into one like we've seen before.

	\begin{problem}
		Use the substitution $z=\frac{y}{x}$ to transform the differential equation $\frac{dy}{dx}=\frac{x^2+3y^2}{ 2xy}, x>0$ into a differential equation in $z$ and $x$. By first solving this new equation, find the general solution of the original equation, giving $y^2$ in terms of $x$.
	\end{problem}

	\begin{columns}
	\begin{column}{.6\linewidth}
	\begin{solution}<2->
		$z=\frac{y}{x} \implies y=xz$

		$\frac{dy}{dx}= x \frac{dz}{dx}+z$ (Product rule)
	\end{solution}
	\begin{solution}<3->
		$x \frac{dz}{dx}+z=\frac{x^2+3x^2z^2}{ 2x^2z}=\frac{1}{2z}+\frac{3}{2}z$
	\end{solution}
	\begin{solution}<4->
		$x \frac{dz}{dx}=\frac{1}{2z}+\frac{1}{2}z=\frac{1+z^2}{2z}$ 

		$\int \frac{2z}{1+z^2} dz=\int \frac{1}{x}dx$ ?

$1+z^2=Ax$

	\end{solution}
	\end{column}
	\begin{column}{.4\linewidth}
		\begin{itemize}
			\item<2-> Make $y$ the subject so that we can find  $\frac{dy}{dx}$ and sub $y$ out.
				\item<3-> Sub in $y$ and  $\frac{dy}{dx}$.
				\item<4-> Simplify to either get a separable case or one with I.F.
				\item<5-> Sub back $y$ in at the end.
		\end{itemize}
	\begin{solution}<5->
		$y^2=x^2(Ax-1)$
	\end{solution}
	\end{column}
	\end{columns}
	
\end{frame}

\begin{frame}{Substitution then Integrating Factor}
	\begin{problem}
		Use the substitution $z=y^{-1}$ to transform the differential equation $\frac{dy}{dx}+xy=xy^2$, into a differential equation in $z$ and $x$. Find the general solution using an integrating factor.
	\end{problem}
	\begin{solution}<2->
		\begin{align*}
			y=\frac{1}{z} &, \frac{dy}{dx}= -\frac{1}{z^2}\frac{dz}{dx} \\
			-\frac{1}{z^2}\frac{dz}{dx}+\frac{x}{z} &= \frac{x}{z^2} \\
			\frac{dz}{dx} -xz &= -x \\
			\text{I.F.} &= e^{\int -x\ dx}=e^{-\frac{1}{2}x^2} \\
			\frac{d}{dx}\left( ze^{-\frac{1}{2}x^2} \right) &= -xe^{-\frac{1}{2}x^2} \\
			ze^{-\frac{1}{2}x^2} &= e^{-\frac{1}{2}x^2}+c \\
			z &= 1+ce^{\frac{1}{2}x^2} \\
			y &= \frac{1}{1+ce^{\frac{1}{2}x^2}}
		.\end{align*}
	\end{solution}
\end{frame}

\begin{frame}[shrink]{Test Your Understanding}
	\begin{problem}
		A particle is moving along the $x$-axis and its displacement, $x$ meters, is modelled using the differential equation 
		\[
		t \frac{dx}{dt} +x = 2t^3x^2, \quad 0<t<1.5
		.\]
		Where $t$ is the time in seconds.
		\begin{itemize}
			\item Use the substitution $u=xt$ to show that the differential equation can be expressed as $\frac{du}{dt}=2u^2t$.
			\item Hence show that the general solution to the differential equation is $ x= \frac{1}{t(A-t^2}$, where $A$ is an arbitrary constant.
			\item Given that $x=1$ when $t=0.5,$ find the displacement after 1.2 seconds.
		\end{itemize}

	\end{problem}
	\begin{columns}[T]
	\begin{column}{.3\linewidth}
	\begin{solution}<2->
		$\frac{du}{dt}=t \frac{dx}{dt}+x$

		$x=\frac{u}{t}$ 

		$t \frac{dx}{dt} + x= 2t^3x^2$ 

		$\frac{du}{dt}=2t^3\left( \frac{u}{t} \right)^2=2u^2t $
	\end{solution}
	\end{column}
	\begin{column}{.3\linewidth}
		\begin{solution}<3->
	$\int \frac{1}{u^2}du=\int 2t dt$

	$-\frac{1}{u}=t^2+c$

	$u=-\frac{1}{t^2+c}$

	$xt=\frac{1}{A-t^2}$

	$x=\frac{1}{t(A-t^2)}$
	\end{solution}
	\end{column}
	\begin{column}{.4\textwidth}
		\begin{solution}<4->
			$x=1$ when $t=0.5$ :

			$1=\frac{1}{0.5(A-0.25)} \implies A =2.25$

			$x=\frac{1}{1.2(2.25-1.2^2}$ 

			The displacement after 1.2s is 1.03m (3s.f)
		\end{solution}
	\end{column}
	\end{columns}
\end{frame}

\begin{frame}[shrink]{Past Paper Question}
	\begin{problem}
		\begin{itemize}
			\item Show that the substitution $y=vx$ transforms the differential equation 
				\[
				3xy^2 \frac{dy}{dx}=x^3+y^3
				.\] 
				Into the differential equation
				\[
				3v^2x \frac{dv}{dx}=1-2v^3
				.\] 
			\item By solving the second differential equation, find a general solution to the first differential equation in the form $y=f(x)$.
			\item Given that $y=2$ at $x=1,$ find the value of $\frac{dy}{dx}$ at $x=1$.
			
		\end{itemize}


	\end{problem}
\begin{columns}[T]
\begin{column}{.5\linewidth}
\begin{solution}<2->
	$\frac{dy}{dx}=v+x \frac{dv}{dx}$

	$3x(vx)^2(v+x \frac{dv}{dx})= x^3+(vx)^3$

	$3(v)^2(v+x \frac{dv}{dx})=1+(v)^3$ 

	$3v^2x \frac{dv}{dx}= 1-2v^3$
\end{solution}
\begin{solution}<4->
         $3(1)(2)^2 \frac{dy}{dx}=(1)^3+(2)^3$
  
          $12 \frac{dy}{dx}=1+8$
  
          $\left.\frac{dy}{dx}\right\rvert_{x=1}=\frac{3}{4}$
  \end{solution}

\end{column}
\begin{column}{.5\linewidth}
\begin{solution}<3->
          $\int \frac{3v^2}{1-2v^3} dv = \int \frac{1}{x} dx$ 
  
          $-\frac{1}{2}\ln (1-2v^3)=\ln x +c$
  
          $1-2v^3=\frac{A}{x^2}$
  
          $x^3-2y^3=Ax$
  
         $y=(\frac{1}{2}(x^3-Ax))^{\frac{1}{3}}$
  \end{solution}
\end{column}
\end{columns}
	
\end{frame}


\end{document}
