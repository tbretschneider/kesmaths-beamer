\documentclass[8pt]{beamer}

 \usepackage[utf8]{inputenc}                                                     
 \usetheme{Bergen}                                                               
%  \usecolortheme{crane}                                                       
  %\useinnertheme{circles}                                                         
  \usepackage[english]{babel}                                                     
  \usepackage{csquotes}                                                           
  \usepackage[T1]{fontenc}                                                        
  \usepackage{booktabs}                                                           
  \usepackage{amsmath}   
  \usepackage{tikz}
  \usepackage{amssymb}
  \usepackage{amsfonts}
  \usepackage{mathrsfs}   
  \usepackage{graphicx}
  \usepackage{varioref}
  \usepackage{probsoln}
  \usepackage[style=authoryear,backref=true]{biblatex}
 \usepackage[]{hyperref} 
  \graphicspath{{Graphics/}}
  \usepackage{multirow,array}
  \addbibresource{../Everything.bib}
  \usepackage{colortbl}
  \definecolor{aa}{RGB}{247, 232, 35}
  \definecolor{cc}{RGB}{29, 23, 80}    
  %\setbeamercolor{palette tertiary}{fg=aa,bg=cc}
  %\setbeamercolor{structure}{fg=cc}
  %\setbeamercolor{alerted text}{fg=red}
  
  %Information to be included in the title page:
  \title[Discrete]{AL FM Discrete}
  
  \subtitle{Binary Operations}
  
  \author[]{T. Bretschneider}
  
  \date[\today]{\today}

\usepackage{comment}
\usepackage{varwidth}

\newcommand{\mat}[4]{\left(\begin{array}{cc} #1 & #2 \\ #3 & #4 \\ \end{array}\right)}
\newcommand{\Q}{\mathbb{Q}}
\newcommand{\R}{\mathbb{R}}
\newcommand{\Z}{\mathbb{Z}}
\newcommand{\sol}[2][+]{
\tikz[baseline]{\node[color=aa,fill=cc,rectangle,draw,anchor=base] {\onslide<#1->{#2}};}
}

\usetikzlibrary{positioning}

\def\height{0.8cm}
\def\width{1.2cm}

		\newcommand{\keynode}[6]{\node[minimum height=\height,minimum width=\width,draw,rectangle,color=aa,fill=cc] (#3) at (#1,#2) {};
	\node[rectangle,minimum height=\height/2,minimum width=\width,above,color=aa,fill=cc] at (#3) {#3};
	\node[draw,rectangle,minimum height=\height/2,minimum width=\width/3,below,color=aa,fill=cc,inner sep =0cm] at (#3) {\footnotesize#4};
	\node[draw,rectangle,minimum height=\height/2,minimum width=\width/3,below,xshift=\height/2,color=aa,fill=cc,inner sep=0cm] at (#3) {\footnotesize#5};
	\node[draw,rectangle,minimum height=\height/2,minimum width=\width/3,below,xshift=-\height/2,color=aa,fill=cc,inner sep=0cm] at (#3) {\footnotesize#6}; }



\begin{document}

\frame{\titlepage}

\begin{frame}
\frametitle{Outline}
\tableofcontents

\end{frame}

\section{Introduction}

\begin{frame}{What is a Binary Operation?}
	\begin{definition}
	  Given a non-empty set $S$, a \textbf{binary operation} on $S$ is a rule for combining any two elements $a,b\in S$ to give a unique result  $c$ where  $c$ is not necessarily an element of  $S$.	
	\end{definition}

	Addition, subtraction and multiplication are all binary operations on $\mathbb{R}$ and division is a binary operation on $\mathbb{R} \backslash \{0\}$.

	Square root and factorial are not binary operations (in fact both of these are unary operations as they only take one input).

	Less familiar binary operations are often defined using a symbol such as  $*,\Delta \text{ or }\odot$.

	\begin{problem}
		Let a binary operation on $\mathbb{Z}$ be defined by $ a* b = a+2b-3$.

		Find 
		\begin{itemize}
			\item $3*5$ = \sol{10}
			\item  $3*0$ = \sol{0}
			\item  $0*3$ = \sol{3}
			\item  $-5*0$ = \sol{-8}
		\end{itemize}
	\end{problem}


\end{frame}

\begin{frame}{Test Your Understanding}
	\begin{problem}
		The Binary operation $a\Delta b = a^2 + b^2 - 2ab$ where  $a,b\in \mathbb{R}$.
		 \begin{itemize}			 
			 \item Find the value of $3\Delta 4$.
			 \item Find the relationship between  $a $ and  $b$ such that $a\Delta b =0$.
		\end{itemize}
	\end{problem}
	\begin{solution}<2->
		$3\Delta 4= 3^2 + 4^2 -2*3*4= 9 + 16 - 24 = 1$
	\end{solution}
\begin{solution}<3->
	$(a-b)^2=0\implies a=b $
\end{solution}
\begin{problem}
	The binary operation $*$ is given by $M*N=MN+M-N$ where  $M$ and  $N$ are  $2\times 2 $ matrices. Show that, for any $M$,  $(M*I)*I=aM+bI$ where $a$ and  $b$ are integers to be determined.
\end{problem}
\begin{solution}<4->
	\begin{align*}
		(M*I)*I &= (MI+M-I)*I \\
			&= (M+M-I)*I \\
			&= (2M-I)*I\\
			&= (2M-I)(I)+(2M-I)-I \\
			&= 2M-I+2M-I-I \\
			&= 4M-3I \text{ so $a=4$ and $b=-3$}
	.\end{align*}

\end{solution}
\end{frame}

\begin{frame}{Closure}
	\begin{definition}
		A binary operation $*$ is said to be closed on a set $S$ if  $a*b\in S \forall a,b \in S$ 
	\end{definition}
	\alert<-1>{Note that closure refers to a set \emph{and} a binary operation}

	The binary operation of addition is closed on $\mathbb{Z}$ as the sum of any two integers is also an integer.

	The binary operation of subtraction is not closed on $\mathbb{N}$ as  $5-7=-2\not\in \mathbb{N}$.

	\begin{problem}
		Determine whether the following binary operations are closed on $\mathbb{Z}.$
		\begin{itemize}
			\item $a*b=\frac{a+b}{a^2}$ 
			\item $a\Delta b = 2^{a+b}$
			\item  $ a \circ b = a+b-3ab$
		\end{itemize}
	\end{problem}

	\begin{solution}<2->
		When $a=2$ and $b=2,2*3=\frac{2\times 3}{4}=\frac{5}{4}\not\in \mathbb{Z}$.	
	\end{solution}
	\begin{solution}<3->
		When $a=-2$ and $b=0,-2\Delta 0=2^{-2+0}=\frac{1}{4}\not\in \mathbb{Z}$
	\end{solution}
	\begin{solution}<4->
		Since $ a$ and $b$ are in $\mathbb{Z}$ then their sum $a+b$ and product $ab$ are also in $\mathbb{Z}$. Therefore $a\circ b$ is closed.
	\end{solution}
\end{frame}

\begin{frame}{Test Your Understanding}
	\begin{problem}
	Which of the following sets are closed under multiplication?
	\begin{itemize}
		\item $\{a+bi|a,b \in \mathbb{Q}, b \neq 0\}$
		\item $\{a+bi|a,b \in \mathbb{Q}, a \neq 0\}$
	\item $\{a+bi|a,b \in \mathbb{Q}\}\backslash  \{0\}$
	\end{itemize}
\end{problem}

\begin{solution}<2->
	The first is not closed because $1+i$ and  $1-i$ are both in the set but  $(1+i)(1-i)=2+0i$ is not.
\end{solution}
\begin{solution}<3->
	The second is not closed because $2+i$ and $1+2i$ are both in the set but $(2+i)(1+2i)=0+5i$ is not.
\end{solution}
\begin{solution}<4->
	If we have two non-zero complex numbers that multiply to give zero: $zw=0$ 

	Then since $|zw|= |z | |w|$, either $|z| \text{ or } |w| =0$.

	Therefore it is impossible to have two numbers in the set multiply to give the only number that isn't.

	So the set is closed under multiplication.
\end{solution}

\end{frame}

\begin{frame}{Test Your Understanding}
	\begin{problem}
	Determine whether the following function are closed on first $ \mathbb{Z}$ and the on $\mathbb{Q}.$
\begin{itemize}
	\item $a*b=a^2-b$
	\item $a \circ b = \frac{a+b}{a}$
	\item $a\Delta b= \sqrt{a^2b^2} $
	\item $a\Omega b=\frac{a+b}{a^2+1}$
\end{itemize}
 \end{problem}
	
 \begin{solution}<2->
Closed on $\Z$ and $\Q$.
 \end{solution}

 \begin{solution}<3->
 	$3\circ 2=\frac{5}{2}\not\in \Z$ so not closed on $\Z$.

	$0\circ 2=\frac{2}{0}\not\in \Q$ so not closed on $\Q$.

 \end{solution}

\begin{solution}<4->
	$\sqrt{a^2b^2}=|ab| .$ If $a,b \in \Z$ then $|ab|\in \Z$ and if $a,b \in \Q$ then $|ab| \in \Q$ so closed on $\Z$ and $\Q$. 
\end{solution}

\begin{solution}<5->
	$1\Omega 2=\frac{3}{2}\not\in \Z$ so not closed on $\Z$.

	Since $a^2+1>0$ then $\frac{a+b}{a^2+1}\in \Q$ so closed on $\Q$.
\end{solution}
\end{frame}

\begin{frame}{Associativity}
	\begin{definition}
		A binary operation $*$ is \textbf{associative} on  a set $S$ if  $a*(b*c)=(a*b)*c \forall a,b,c \in S$.
	\end{definition}

	\alert<1>{The for all in the definition makes it easier to show that a binary operation is not associative as you only have to provide one counter example.}

	\begin{problem}
		Determine whether the following binary operations on $\R$ are associative:
		\begin{itemize}
			\item $a*b=2a+3b$
			\item  $a \circ b = a + b + ab$
		\end{itemize}
	\end{problem}

	\begin{solution}<2->
		$(1*0)*2=2*2=10$ whereas  $1*(0*2)=1*6=20$.

		Therefore $*$ in not associative on  $ \R$.

	\end{solution}
	\begin{solution}<3->
		\begin{align*}
			(a \circ b)\circ c &= (a+b+ab)\circ c \\
					   &= a+b+ab+c+(a+b+ab)(c) \\
					   &= a +b+ab+c+ac+bc+abc 
		.\end{align*}
		\begin{align*}
			a \circ(b\circ c) &= a \circ (b+c+bc) \\
					  &= a+b+c+bc+a(b+c+bc) \\
					  &= a+b+c+bc+ab+ac+abc 
		.\end{align*}
$\implies \circ$ is associative.

	\end{solution}
\end{frame}

\begin{frame}{Past Paper Question}
	\begin{problem}
		The function $\min (a,b)$ is defined by:  \[
			\min(a,b)= \begin{cases}
				a, & a < b \\
				b, & \text{otherwise.}
			\end{cases}
		\]

		Gary claims that the binary operation $\Delta$, which is defined as  \[
			x \Delta y = \min(x,y-3)
		\] 
		where $x$ and  $y$ are real numbers, is associative as finding the smallest number is not affect by the order of the operation.

		Disprove Gary's claim.
	\end{problem}
	\begin{solution}<2->
		$2\Delta(1\Delta 3)=2\Delta (0) = -3 \neq (-2)\Delta 3=(2\Delta 1) \Delta 3. \qedsymbol$ 
	\end{solution}
\end{frame}

\begin{frame}[shrink]{Commutativity}
	\begin{definition}
		A binary operation $*$ is said to be commutative on a set  $S$ if  $a*b=b*a \forall a,b \in S$.
	\end{definition}
	\begin{problem}
		Determine whether the following operations on $\R$ are commutative.
		\begin{itemize}
			\item $a*b=2a+b$
			\item  $a\Delta b = 3^{a+b}$.
		\end{itemize}
	\end{problem}
	\begin{solution}<2->
		$3*2=2\times 3 +2 =8$ whereas $2*3= 2\times 2+3=7$

		So $*$ is not commutative on  $\R$
	\end{solution}
	\begin{solution}<3->
		\begin{align*}
			a\Delta b &= 3^{a+b} \\
				  &= 3^{b+a} \text{ (since addition is commutative on $ \R$)} \\
				  &= b\Delta a 
		.\end{align*}
So $\Delta$ is commutative on  $\R$.

	\end{solution}

	\begin{problem}
		If $*$ is both associative and commutative on a set  $S$, show that  $(a*b)^2=a^2*b^2.$
	\end{problem}

	\begin{solution}<4->
		\begin{align*}
			(a*b)^2&=(a*b)*(a*b)\\
			       &= a*(b*a)*b \text{ (because $*$ is associative on  $S$)} \\
			       &= a*(a*b)*b \text{ (because $*$ is commutative on  $S$)} \\
			       &= (a*a)*(b*b) \text{ (because $*$ is associative on  $S$)} \\
			       &= a^2*b^2  
		.\end{align*}
		
	\end{solution}
\end{frame}

\begin{frame}{Past Paper Question}
	\begin{problem}
		The binary operation $\blacklozenge$ is defined as  $a\blacklozenge b= a^{b}$ where $a$ and  $ b$ are non-zero real numbers.
		 \begin{itemize}
			\item Determine whether or not $\blacklozenge$ is associative.
\item Tim claims that as $2\blacklozenge 4= 4 \blacklozenge 2 $ then $\blacklozenge$ is commutative.

	Assess the validity of Tim's claim.
		\end{itemize}
	\end{problem}

	\begin{solution}<2->
		It is not associative. $a\blacklozenge (b\blacklozenge c) = a^{(b^{c})} \neq a^{bc} = (a^{b})^{c}=(a\blacklozenge b)\blacklozenge c$
	\end{solution}

	\begin{solution}<3->
		He has only checked it for one case but commutative must hold for all $a,b\in \R$. It is also not commutative since $1\blacklozenge 2 = 1 \neq 2 = 2 \blacklozenge 1.$
	\end{solution}
\end{frame}

\begin{frame}[shrink]{Identity}
	\begin{definition}
		For a binary operation $*$ on a set  $S$, if there exists an element $e$ such that $x*e=e*x=x$ for all  $x\in S$ then $e $ is called the \textbf{identity} element. 
	\end{definition}

	\alert<1>{This does not mean $*$ has to be commutative on  $S$, just that  $e$ commutates.}

\begin{problem}
	For the following binary operations, determine whether an identity element exists in $\R$:
	\begin{itemize}
		\item $a*b= 3ab$
		\item  $ a \circ b = 3a+b$
	\end{itemize}

\end{problem}

\begin{solution}<2->
	Using the definition we have that: 
	\begin{align*}
		a*e=3ae&=a \\
		(3e-1)(a)&=0 \\
		e&=\frac{1}{3}
	.\end{align*}

	Easy to check that this also works for $e*a$. So an identity exists for  $*$ on  $\R$ and is $\frac{1}{3}$.
\end{solution}

\begin{solution}<3->
	\begin{align*}
		a\circ e = 3a+e&= a \\
		e &= -2a 
	.\end{align*}
	Since the identity element is constant it cannont change depending on $a$ so no identity for  $\circ$ on  $\R$ exists.
\end{solution}


\end{frame}

\begin{frame}{Inverse}
	\begin{definition}
		Consider a binary operation $*$ on a set  $S$ with an identity element  $e$. For  $x\in S$, an inverse element  $x^{-1}\in S$ exists if and only if $x*x^{-1}=x^{-1}*x=e$.
	\end{definition}

	\alert<1>{Note that this definition does not make sense unless there is an identity element.}

	For addition in $\R$, the inverse of the element $a$ is \sol{ $-a$}.

	For addition on the set $\{0,1,2,3,\ldots\}$ there is no inverse for any element apart from 0 (even though there is an identity).

	For multiplication on $\R \backslash \{0\}$ the inverse of an element $a$ is \sol{ $\frac{1}{a}$ }.

	\begin{problem}
		The binary operation $\lozenge$ is given by  $a\lozenge b = a+b-ab$ where  $a,b\is \Z$.
		\begin{itemize}
			\item Show that the identity element in the set $\Z$ with respect to the binary operation $\lozenge$ is 0.
			\item Find the inverse of the element 4 under the binary operation  $\lozenge$.
		\end{itemize}

	\end{problem}
	\begin{solution}<3->
		$a\lozenge 0 = a+0 -a \times 0=a $ and $0\lozenge a=0+a-0\times a=a$, so 0 is the identity element.
	\end{solution}
	\begin{solution}<4->
		If $a\lozenge 4=0$ then $a+4-4a=0$ and  $a=\frac{4}{3}$. This $4^{-1}=\frac{4}{3}$. 
	\end{solution}
	
\end{frame}

\begin{frame}{Cayley Tables}
	\begin{definition}
		A \textbf{Cayley table} is a square array where the elements of a set $S$  are the headings of the rows and columns and the elements are the result of inputting the row header and column header into a binary operation $*$. 
	\end{definition}
	\begin{problem}
		Let a binary operation $*$ on a set  $\{0,1,2,3\}$ be defined by  $a*b=a^2+ab$.
		\begin{itemize}
			\item Construct the Cayley table for $*$.
			\item Is the operation  $*$ closed on  $S$?
			\item Is the operation  $*$ commutative?
		\end{itemize}
	\end{problem}
	\begin{solution}<2->
		\begin{center}
			\colorbox{cc!30}{
			\setlength\arrayrulewidth{0.5mm}
			\arrayrulecolor{white}
			\begin{tabular}{c|cccc}
				$*$ & 0 & 1 & 2 & 3 \\
				\hline
				0 & 0 & 0 & 0 & 0 \\
				1 & 0 & 2 & 3 & 4 \\
				2 & 0 & 6 & 8 & 10 \\
				3 & 0 & 12 & 15 & 18 \\
			\end{tabular}
		}
		\end{center}
	\end{solution}
	\begin{solution}<3->
		The operation $*$ is clearly not closed on  $S$. There are elements in the Cayley table which are not in  $S$. Take  $1*3=4\not\in S$.
	\end{solution}
	\begin{solution}<4->
		The lack of symmetry around the leading diagonal indicates that the operation $*$ is not commutative.
	\end{solution}
\end{frame}

\begin{frame}{Test Your Understanding}
	\begin{problem}
		Let $U_{4}=\{1,i,-i,-1\}$ and consider the operation of multiplication it $ U_4$.
		\begin{itemize}
			\item  Construct the Cayley table for the operation.
			\item Is the operation commutative or associative?
			\item State the identity if it exists.
			\item Find the inverse of each element, if possible.

		\end{itemize}
	\end{problem}
\begin{solution}<2->
	
		\begin{center}
			\colorbox{cc!30}{
			\setlength\arrayrulewidth{0.5mm}
			\arrayrulecolor{white}
			\begin{tabular}{c|cccc}
				$\times $ & $1$ & $i$ & $-i$ & $-1$ \\
				\hline
				$1$ & $1$ & $-i$ & $i$ & $-1$ \\
				$i$ & $i$ & $-1$ & $1$ & $-i$ \\
				$-i$ $-i$  & $1$ & $-1$ & $i$ \\
				$-1$ & $-1$ & $-i$ & $i$ & $1$ \\
			\end{tabular}
		}
		\end{center}
\end{solution}	
\begin{solution}<3->
	The operation is both commutative and associative as multiplication is associative and commutative on the parent group of $\mathbb{C}.$
\end{solution}
\begin{solution}<4->
	The identity is 1.
	
\end{solution}
\begin{solution}<5->
	1 and  $-1$ are both self inverse. $i$ and  $-i$ are inverse pairs.
\end{solution}
\end{frame}

\begin{frame}{Modular Arithmetic}
	\begin{definition}
		\textbf{Modular arithmetic} is a system of arithmetic for integers where numbers 'wrap around' upon reaching a certain value called the modulus.
	\end{definition}
	\begin{definition}
		$+_n$ means addition modulo  $n$.

		 $\times_n$ means multiplication modulo $n$.
	\end{definition}

	You do modular arithmetic all the time when reading a clock.

	10 o'clock plus three hours is 1 o'clock because we wrap around at 12.

	We would write $10+3\equiv 1 \mod 12$ or  $10 +_{12} 3 =1$.

	$5 +_{12} 10 = $\sol{$3$},  $5\times_{12} 10=$\sol{2}, $6\times_9 3=$\sol{0}, $8+_{10} 13=$\sol{1},  $1+_{10}1=$\sol{1}.

	 \begin{problem}
		 The binary operation $\blacksquare$ is given by  $x\blacksquare y=x^2+y+2 \mod 16$ where $x,y \in \Z$.
		 \begin{itemize}
		 	\item Find the value of $5\blacksquare 3$.
			\item Find two integers  $a$ and $b$ such that $a\blacksquare b=0$.
		 \end{itemize}

	\end{problem}

	\begin{solution}<6->
		$5\blacksquare 3=5^2+3+2 \mod 16 = 30 \mod 16=14$.
	\end{solution}

	\begin{solution}<7->
		$a^2+b=14, a=3, b=5$
	\end{solution}
	
\end{frame}

\begin{frame}[shrink]{Test Your Understanding}
	\begin{problem}
		The binary operation $*$ is defined as  $x*y=x+y+1 \mod 2$ where  $x,y\in \R$.
		\begin{itemize}
			\item Prove that the binary operation $*$ is associative.
			\item Find an identity element of the set $ \R$ with respect to the binary operation $*$.
			\item Prove that the sat $ \R$ has infinitely many identity elements with respect to $*$.

		\end{itemize}
	\end{problem}

\begin{solution}<2->
	\begin{align*}
		(a*b)*c &= (a+b+1 \mod 2) * c \\
		&= a+b+1+c+1 \mod 2 \\
		&= a + b +c \mod 2 
	.\end{align*}
	\begin{align*}
		a*(b*c) &= a*(b+c+1 \mod 2) \\
		&= a+b+c+1+1 \mod 2 \\
		&= a+b+c \mod 2 
	.\end{align*}
	$\therefore a*(b*c)=(a*b)*c$ for all  $a,b,c \in \R$ and so $*$ is associative.
\end{solution}
\begin{solution}<3->
	For the identity element $e, e*a=a*e=a=a+e+1 \mod 2$.

	So $e+1=0 \mod 2$, and so  $e=1$ works.

	So 1 is an identity element.
\end{solution}
\begin{solution}<4->
	Anything equal to $1 \mod 2$ is an identity element. So anything of the form  $1+2k, \forall k \in \Z$, will work. This is identical to all odd numbers.
\end{solution}

\end{frame}

\begin{frame}[shrink]{Test Your Understanding}
	\begin{problem}
		$S=\{1,2,3,4,5,6,7,8,9,10,11\}$. Find the inverse of the element  $5$ under multiplication  $\mod 12$.
	\end{problem}
	\begin{solution}<2->
		$5\times_{12} 5=1 $ so $ 5^{-1}=5$. It is self inverse.
	\end{solution}
	\begin{problem}
		\begin{itemize}
			\item Construct a Cayley table far multiplication modulo 5 on $\{1,2,3,4\}.$
			\item Use the table to solve the following:
				 \begin{itemize}
					\item $2x=1 \mod 5$
					\item  $4x+3=4 \mod 5$
				\end{itemize}
			\item State the identity if it exists.
			\item Verify that $3^{-1}=2$.
			\item State the inverse of each other element.
		\end{itemize}
	
	\end{problem}
	\begin{solution}<3->
		
		\begin{center}
			\colorbox{cc!30}{
			\setlength\arrayrulewidth{0.5mm}
			\arrayrulecolor{white}
			\begin{tabular}{c|cccc}
				$\times_5 $ & $1$ & $2$ & $3$ & $4$ \\
				\hline
				$1$ & $1$ & $2$ & $3$ & $4$ \\
				$2$ & $2$ & $4$ & $1$ & $3$ \\
				$3$ & $3$ & $1$ & $4$ & $2$ \\
				$4$ & $4$ & $3$ & $2$ & $1$ \\
			\end{tabular}
		}
	
	\end{center}
	
\end{solution}
\begin{solution}<4->
	$x=3$
\end{solution}
\begin{solution}<5->
	$4x=1\mod 5 \implies x=4$
\end{solution}
\begin{solution}<6->
	The identity is 1.
\end{solution}
\begin{solution}<7->
	$3\times_5 2=2\times_5 3 =1$
\end{solution}
\begin{solution}<8->
	1 and 4 are self inverse. 2 and 3 are an inverse pair.
\end{solution}

	\end{frame}


\end{document}
