% !TeX spellcheck = en_B
% !TeX encoding = UTF-8
\documentclass[8pt]{beamer}
\tracinggroups=1

\usepackage[utf8]{inputenc}                                                     
 \usetheme[block=fill,progressbar=foot,background=light]{metropolis}                                                               
%  \usecolortheme{crane}                                                       
  %\useinnertheme{circles}                                                         
  \usepackage[english]{babel}                                                     
  \usepackage{csquotes}                                                           
  \usepackage[T1]{fontenc}                                                        
  \usepackage{booktabs}                                                       \usepackage{pgfgantt}
  \usepackage{pifont}
  \usepackage{adfbullets}
  \usepackage{enumitem}
  \usepackage{amsmath}   
  \usepackage{tikz}
  \usepackage{amssymb}
  \usepackage{amsfonts}
  \usepackage{mathrsfs}   
  \usepackage{graphicx}
  \usepackage{adjustbox}
  \usepackage{varioref}
  \usepackage{probsoln}
  \usepackage{attachfile2}
  \usepackage{pgfplots}
\pgfplotsset{compat=newest}
  \usepackage[style=authoryear,backref=true]{biblatex}
 \usepackage[]{hyperref} 
  \graphicspath{{Graphics/}}
  \usepackage{multirow,array}
  \addbibresource{../Everything.bib}
  \usepackage{colortbl}
  \definecolor{aa}{RGB}{255, 124, 0}
  \definecolor{cc}{RGB}{230, 230, 230}    
  %\setbeamercolor{palette tertiary}{fg=aa,bg=cc}
  %\setbeamercolor{structure}{fg=cc}
  %\setbeamercolor{alerted text}{fg=red}
  
  %Information to be included in the title page:
  
  \usebackgroundtemplate{%
  \tikz[overlay,remember picture]{\node[scale=80,opacity=0.03, at=(current page.south east)] {\adfbullet{9}};}}
  
  \author[]{T. Bretschneider}
  
  \date[\today]{\today}

\usepackage{comment}
\usepackage{varwidth}

\newcommand{\mat}[4]{\left(\begin{array}{cc} #1 & #2 \\ #3 & #4 \\ \end{array}\right)}
\newcommand{\Q}{\mathbb{Q}}
\newcommand{\R}{\mathbb{R}}
\newcommand{\Z}{\mathbb{Z}}
\newcommand{\sol}[2][+]{
	\tikz[baseline]{\node[color=aa,fill=cc,rectangle,draw,anchor=base] {  {\onslide<#1->{#2}}  };}
}

\usetikzlibrary{positioning}
\usetikzlibrary{tikzmark}
\usetikzlibrary{shadings}
\usetikzlibrary{through}


\def\height{0.8cm}
\def\width{1.2cm}

		\newcommand{\keynode}[6]{\node[minimum height=\height,minimum width=\width,draw,rectangle,color=aa,fill=cc] (#3) at (#1,#2) {};
	\node[rectangle,minimum height=\height/2,minimum width=\width,above,color=aa] at (#3) {#3};
	\node[draw,rectangle,minimum height=\height/2,minimum width=\width/3,below,color=aa,fill=cc,inner sep =0cm] at (#3) {\footnotesize#4};
	\node[draw,rectangle,minimum height=\height/2,minimum width=\width/3,below,xshift=\height/2,color=aa,fill=cc,inner sep=0cm] at (#3) {\footnotesize#5};
	\node[draw,rectangle,minimum height=\height/2,minimum width=\width/3,below,xshift=-\height/2,color=aa,fill=cc,inner sep=0cm] at (#3) {\footnotesize#6}; }

\newenvironment{gantt}[3]{\begin{ganttchart}[#1,bar height=.6,bar top shift=.2,title/.style=  {draw=none},y unit chart=0.6cm,y unit title = 0.6cm,include title in canvas=false,group/.append style={draw=black,dashed},bar/.append style={fill=aa},inline,hgrid=true,Float1/.style={bar/.append style={fill=none,dashed},bar height=.8,bar top shift=0.1}]{#2}{#3}}{\end{ganttchart}}

\newenvironment{nicetable}[1]{\setlength\arrayrulewidth{0.5mm}
			\arrayrulecolor{white}
			\begin{tabular}{#1}}{\end{tabular}}
		
\setlist[itemize,1]{label={\color{aa}\huge\adfbullet{9}}}
\setlist[itemize, 2]{label={\color{aa}\large\adfbullet{9}}}

\newcommand\reshist{}
\def\reshist(#1)#2(#3)#4(#5)%
{\draw (axis cs:#1) rectangle (axis cs:#3) node [midway] {#5};}






\title[Pure]{{\color{aa}\Huge\adfbullet{9}}AL FM Discrete}
\subtitle{LP in Game Theory, \textattachfile{LPinGameTheory.tex}{(TeX)}}

\begin{document}

\frame{\titlepage}

\setlength{\abovedisplayskip}{0pt}
\setlength{\belowdisplayskip}{0pt}
\setlength{\abovedisplayshortskip}{0pt}
\setlength{\belowdisplayshortskip}{0pt}

\begin{frame}{Motivation}
	We have so far come across the following methods for finding the optimal mixed strategy from a pay-off matrix:
\begin{itemize}
	\item Solving simultaneous equations ($2 \times 2$)
	\item Drawing the graphs and find the maxmin point  ($2 \times 2$)
    \item By rewriting a $n \times 2$ pay-off in terms of the other person to reducing the problem.
\end{itemize}
Large pay-off matrices can be solved if there are dominated strategies allowing reduction to one of the smaller sizes above.

\alert{We want a method that can be extended to larger matrices and gives a programmatic way of solving the problems. We can formulate the problem of finding the largest minimum pay-off of the expected pay-off equations as a linear programming problem. }

\end{frame}

\begin{frame}{Formulation of the LP}
	We start with showing the method for a $2\times 2$.

	\begin{problem}
	\begin{minipage}{.5\linewidth}
		\begin{itemize}
		\item Formulate as an LP
		\item Use the simplex method to solve
	\end{itemize}
\end{minipage}%
\begin{minipage}{.3\linewidth}
	\begin{center}
	\colorbox{cc!30}{
	\begin{nicetable}{cc|cc}
\multicolumn{2}{c}{} & \multicolumn{2}{c}{Player 2}\\
\multicolumn{1}{c}{} &  & $X$  & $Y$ \\ \cline{2-4} 
\raisebox{0.4cm}{\multirow{2}*{\rotatebox{90}{Player 1}}}  & $P$ & $3$ & $-2$ \\
& $Q$ & $1$ & $4$ \\
\end{nicetable}}
\end{center}
\end{minipage}
\end{problem}

\begin{exampleblock}{Step 1}
	All the entries have to be positive. So add the smallest possible value to every entry so that they all become positive.
		\begin{center}
			\colorbox{cc!30}{
	\begin{nicetable}{cc|cc}
\multicolumn{2}{c}{} & \multicolumn{2}{c}{Player 2}\\
\multicolumn{1}{c}{} &  & $X$  & $Y$ \\ \cline{2-4} 
\raisebox{0.4cm}{\multirow{2}*{\rotatebox{90}{Player 1}}}  & $P$ & $6$ & $1$ \\
& $Q$ & $4$ & $7$ \\
\end{nicetable}}
\end{center}
We have added 3 to each value. We do this so that we can write an inequality with probabilities summing to 1.
\end{exampleblock}
\end{frame}

\begin{frame}{Formulating the LP}

\begin{exampleblock}{Step 2}
	We write down the objective function. This is what we want to maximise.
	Let $V$ be the value of the new matrix.  This implies that $V-3$ is the value of the original game, which is what we want to maximise. Hence this is also the objective function. $P=V-3$.
	
\end{exampleblock}

\begin{exampleblock}{Step 3}
	Constraints come from the expected pay-offs being larger \emph{or equal} to the value and the probabilities summing to 1. In the sense of an inequality being less than or equal to 1. However since all entries in the matrix are positive we know that increasing the probabilities will always increase the value, hence this the inequality suffices.  
	
Expected pay-off if Player 2 plays strategy $X$ is  $6p+4q$. Thus  $V\leq 6p+4q$. Hence  $V-6p-4q\leq 0$.

And, expected pay-off if Player 2 plays strategy $Y$ is  $1p+7q$. Hence  $V-p-7q\leq 0$. 

The final constraint is simply $p+q \leq 1$. Here we would expect the slack variable to always be 0 though since all entries in the matrix are positive. 
\end{exampleblock}
\end{frame}


\begin{frame}[allowframebreaks]{Rewriting for the Simplex algorithm}
	Rewriting everything using slack variables leads to:

	\begin{flalign*}
		\text{Maximise} && P &= V-3 && \\
		\text{Subject to} && V -6p-4q + s_1 &= 0 && \\
				  && V -p-7q + s_2 &= 0 && \\
				  && p + q + s_3 &= 1 &&
	.\end{flalign*}


This leads to the following initial simplex tableau:

\begin{center}
\colorbox{cc!30}{
\begin{nicetable}{c|cccccc|c}
$P$ & $v$ & $p$ & $q$ & $s_1$ & $s_2$ & $s_3$ & RHS \\ 
  \hline
1 & $-1$ & $0$ & $0$ & 0 & 0 & 0 & $-3$   \\ 
   \hline
0 & 1 & $-6$ & $-4$ & 1 & 0 & 0 & 0 \\ 
  0 & 1 & $-1$ & $-7$ & 0 & 1 & 0 & 1  \\ 
  0 & 0 & $1$ & 1 & 0 & 0 & 1 & 1 \\ 
\end{nicetable}}
\end{center}

\begin{center}
\colorbox{cc!30}{
\begin{nicetable}{c|cccccc|c}
$P$ & $v$ & $p$ & $q$ & $s_1$ & $s_2$ & $s_3$ & RHS \\ 
  \hline
1 &\boxed{$-1$} & $0$ & $0$ & 0 & 0 & 0 & $-3$   \\ 
   \hline
0 & 1 & $-6$ & $-4$ & 1 & 0 & 0 & 0 \\ 
  0 & 1 & $-1$ & $-7$ & 0 & 1 & 0 & 0  \\ 
  0 & 0 & $1$ & 1 & 0 & 0 & 1 & 1 \\ 
\end{nicetable}}
\end{center}

Using the indicated pivot element our second iteration is:

\begin{center}
\colorbox{cc!30}{
\begin{nicetable}{c|cccccc|c}
$P$ & $v$ & $p$ & $q$ & $s_1$ & $s_2$ & $s_3$ & RHS \\ 
  \hline
1 & $0$ & $-6$ & $-4$ & 1 & 0 & 0 & $-3$   \\ 
   \hline
0 & 1 & $-6$ & $-4$ & 1 & 0 & 0 & 0 \\ 
0 & 0 & \boxed{5} & $-3$ & $-1$ & 1  & 0 & 0  \\ 
  0 & 0 & $1$ & 1 & 0 & 0 & 1 & 1 \\ 
\end{nicetable}}
\end{center}

Using the newly indicated pivot we get for the third iteration (remember the same row can't be used again.):

\begin{center}
\colorbox{cc!30}{
\begin{nicetable}{c|cccccc|c}
$P$ & $v$ & $p$ & $q$ & $s_1$ & $s_2$ & $s_3$ & RHS \\ 
  \hline
1 & $0$ & $0$ & $-7.6$ & $-0.2$ & $1.2$ & 0 & $-3$   \\ 
   \hline
0 & 1 & $0$ & $-7.6$ & $-0.2$ & $1.2$ & 0 & 0 \\ 
  0 & 0 & $1$ & $-0.6$ & $-0.2$ & $0.2$ & 0 & 0  \\ 
  0 & 0 & $0$ & \boxed{$1.6$} & $0.2$ & $-0.2$ & 1 & 1 \\ 
\end{nicetable}}
\end{center}

Finally for the fourth and final iteration we get:

\begin{center}
\colorbox{cc!30}{
\begin{nicetable}{c|cccccc|c}
$P$ & $v$ & $p$ & $q$ & $s_1$ & $s_2$ & $s_3$ & RHS \\ 
  \hline
1 & $0$ & $0$ & $0$ & $0.75$ & $0.25$ & $4.75$ & $1.75$   \\ 
   \hline
0 & 1 & $0$ & $0$ & $0.75$ & $0.25$ & $0.45$ & $4.75$ \\ 
  0 & 0 & $1$ & $0$ & $-0.125$ & $0.125$ & $0.375$ & $0.375$  \\ 
  0 & 0 & $0$ & 1 & $0.125$ & $-0.125$ & $0.625$ & $0.625$ \\ 
\end{nicetable}}
\end{center}

The table is now optimal, giving $P=1.75$.

\end{frame}


\begin{frame}[allowframebreaks]{Formulation with a bigger matrix}
	
\begin{problem}
		For the game shown, formulate the game as a linear programming problem to find the optimal mixed strategy for Player 1.
\end{problem}

		\begin{center}
\colorbox{cc!30}{
\begin{nicetable}{cc|ccc}
\multicolumn{2}{c}{} & \multicolumn{3}{c}{Player $2$}\\
\multicolumn{1}{c}{} &  & $X$  & $Y$ & $Z$ \\ \cline{2-5}
\raisebox{0.0cm}{\multirow{3}*{\rotatebox{90}{Player $1$}}}  & $P$ & $4$ & $2$ & $2$ \\
& $Q$ & $-3$ & $5$ & $1$ \\
& $R$ & $2$ & $-1$ & $3$ \\
\end{nicetable}}
\end{center}
	

	First make all values positive by adding 4. 
	\begin{center}                                        
  \colorbox{cc!30}{                                                   
  \begin{nicetable}{cc|ccc}                                       
  \multicolumn{2}{c}{} & \multicolumn{3}{c}{Player $2$}\\       
  \multicolumni{1}{c}{} &  & $X$  & $Y$ & $Z$ \\ \cline{2-5}        
  \raisebox{0.0cm}{\multirow{3}*{\rotatebox{90}{Player $1$}}}  & $P$ & $  8$ & $6$ & $6$ \\
  & $Q$ & $1$ & $9$ & $5$ \\                                           
  & $R$ & $6$ & $3$ & $7$ \\                                           
  \end{nicetable}}                                                        
  \end{center}
Let each Player 1 pick their strategies with the lower-case letter probabilities.

So for player 2 choosing:
\begin{flalign*}
	X; &&  \text{pay-off} &= 8p+1q+6r && \\
	Y; && &= 6p+9q+3r && \\
	Z; && &= 6p+5q+7r.  &&
\end{flalign*}

\begin{flalign*}
                  \text{Maximise} && P &= V-4 && \\
                  \text{Subject to} && V -8p-q-6r + s_1 &= 0 && \\
                                    && V -6p-9q -3r+ s_2 &= 0 && \\
				    && V -6p-5q-7r+s_3 &=0 && \\
                                    && p + q +r +s_4 &= 1 &&
  \end{flalign*} 

\end{frame}

\begin{frame}[shrink=10]{Past Paper Question}
	\begin{columns}[T]
	\begin{column}{.5\linewidth}
	\begin{problem}
		Harry and Tom play a zero-sum game. The game is represented by the following pay-off matrix for Harry.

		\begin{center}                             
    \colorbox{cc!30}{                              
    \begin{nicetable}{cc|ccc}                      
    \multicolumn{2}{c}{} & \multicolumn{3}{c}{Tom}\\
    \multicolumn{1}{c}{} &  & $X$  & $Y$ & $Z$ \\ \cline{2-5}
    \raisebox{0.0cm}{\multirow{3}*{\rotatebox{90}{Harry}}}  & $A$ & $-3$ & $2$ & $3$ \\
    & $B$ & $-2$ & $3$ & $2$ \\                                           
    & $C$ & $2$ & $-1$ & $-1$ \\                                           
    \end{nicetable}}                                         
    \end{center}

    The game does not have a stable solution. Harry wants to know the optimal mixed strategy he should play in order to maximise his winnings from the game. He begins be defining the following variables: 
    \begin{align*}
	    p_1 &= \text{the probability of Harry playing strategy $A$} \\ 
	    p_2 &= \text{the probability of Harry playing strategy $B$} \\ 
	    p_3 &= \text{the probability of Harry playing strategy $C$}  
    .\end{align*}
	\end{problem}
	\end{column}
	\begin{column}{.5\linewidth}
	\begin{problem}
		\begin{itemize}
			\item Formulate Harry's situation as a linear programming problem.
			\item Harry solves the linear programming problem and finds:
				\begin{align*}
					p_1 &= 0.1875 \\
					p_2 &= 0.1875 \\
					p_3 &= 0.625 \\
				.\end{align*}

				Find the value of the game for Tom. Fully justify your answer.
			\item Tom tells Harry that he is only going to play strategy $X$. Explain which strategy or strategies Harry should play.
		\end{itemize}
	\end{problem}
	\end{column}
	\end{columns}	
\end{frame}

\begin{frame}{Past Paper Solution}
	\begin{columns}
	\begin{column}{.5\linewidth}
	\begin{solution}<2->
		There are no dominated strategies.

		 \[
			 \left( \begin{matrix}
					 1 & 6 & 7 \\
					 2 & 7 & 6 \\
					 6 & 3 & 3 \\
			 \end{matrix} \right) 
		.\]

		\begin{align*}
			p_1+2p_2+6p_3 &\geq  v \\
			6p_1+7p_2+3p_3 &\geq  v \\
			7p_1+6p_2+3p_3 &\geq  v \\
			p_1+p_2+p_3 &\leq  1 \\
			p_1,p_2,p_3 &\geq  v 
		.\end{align*}


		\begin{flalign*}
			\text{Maximise} && v-4 & && \\
			\text{Subject to} && p_1+2p_2+6p_3 &\geq  v && \\
					  && 6p_1+7p_2+3p_3 &\geq  v && \\
					  && 7p_1+6p_2+3p_3 &\geq  v && \\
					  && p_1+p_2+p_3 &\leq  1 && \\
					  && p_1,p_2,p_3 &\geq  v . && 
		\end{flalign*}
	\end{solution}
	\end{column}
	\begin{column}{.5\linewidth}
	\begin{solution}<3->
		$ p_1 +2p_2+6p_3 \\ = 0.1875 + 2\times 0.1875 + 6 \times 0.625 \\ [=4.3125]$

		$=4.3125-4$

		 $=0.3125$

		 The value of the game for Tom is  $-0.3125$ because Harry's winnings  $+$ Tom's winnings  $=$ 0 since they are playing a zero-sum game.
	\end{solution}
	\begin{solution}<4->
		Harry should only play strategy $C$ each time, because this will result in Harry winning 2 each game.

		Playing strategy  $A$ or $B$ will result in Harry losing 3 or 2 each game.
	\end{solution}
	\end{column}
	\end{columns}
\end{frame}
\end{document}
