% !TeX spellcheck = en_GB
% !TeX encoding = UTF-8
\documentclass[8pt]{beamer}

 /home/tobias/kesmaths-beamer/tex/preamble.tex


  \title[Pure]{{\color{aa}\Huge\adfbullet{9}}AL FM Pure}
  \subtitle{Complex Roots of Equations, \textattachfile{ComplexRootsofEquations.tex}{(TeX)}}

\begin{document}

\setlength{\abovedisplayskip}{0pt}
\setlength{\belowdisplayskip}{0pt}
\setlength{\abovedisplayshortskip}{0pt}
\setlength{\belowdisplayshortskip}{0pt}


\frame{\titlepage}

\begin{frame}[shrink=10]{Roots in Conjugate Pairs}
	\begin{definition}
		Complex roots of polynomial equations with real coefficients always occur in complex conjugate pairs.
In other words, if $a+bi$ is the root of an equation then so is $a-bi$.
	\end{definition}

	\alert{Note that this is only true if the coefficients of the polynomial are real.}

	\begin{problem}
		Prove that complex roots of a quadratic, with real coefficients, come in conjugate pairs.
	\end{problem}

	\begin{columns}[T]
	\begin{column}{.5\linewidth}
		\textbf{Proof 1:}
Using the quadratic formula:
		\begin{solution}<2->
			\[
			x=\frac{-b \pm \sqrt{b^2+4ac} }{2a}.
			.\] 

			We can see the real part (before the $\pm$) is the same (provided that  $b,a$ are real), so if one root is complex, the other is its conjugate.
		\end{solution}
	\end{column}
	\begin{column}{.5\linewidth}
	\textbf{Proof 2:}
Let the roots be $\alpha$ and $\beta$.
	Then $(x-\alpha)(x-\beta)=$\sol{$x^2-(\beta+\alpha)x+\alpha\beta$}.
	Since the statement says that there are real coefficients, $\alpha + \beta$ and  $\alpha \beta$ must be real.
	Let  $\alpha=a_1+b_1 i$, and $\beta=a_2 + b_2 i$.
	Then $\alpha +\beta=$\sol{$a_1 + a_2 + (b_1 +b_2)i$.}
	If we require this to be real then \sol{$ b_1 +b_2=0$} and therefore $b_2=-b_1$.
	Also $\alpha \beta= $\sol{$(a_1+ b_1 i)(a_2+ b_2 i)$}\sol{$= a_1 a_2-b_1 b_2+(a_1b_2+a_2b_1)i$.}
	In order for this to be real, $ a_1b_2+a_2b_1=0.$
	But $b_2=-b_1$ therefore: 
	$-a_1b_1+a_2b_1=0$ and $b_1(a_2-a_1)=0$
	Thus either $b_1=b_2=0$(i.e the roots weren't complex) or $ a_2=a_1$, which combined with $b_2=-b_1$, means the roots are conjugates.

	\end{column}
	\end{columns}
\end{frame}

\begin{frame}{Test Your Understanding}
	\begin{problem}
		Given that $\alpha=7+2i$ is one of the roots of a monic quadratic equation with real coefficients,
		 \begin{itemize}
			\item state the value of the other root, $\beta$.
			\item find the quadratic equation.
		\end{itemize}
	\end{problem}

	\begin{solution}<2->
		$\beta=7-2i$
	\end{solution}

	\begin{solution}<3->
		\textbf{Slow way:} 
		\[
			(z-(7-2i))(z-(7+2i))=0
		.\] 
		\[
			z^2-(7+2i)z-(7-2i)z+(7-2i)(7+2i)=0
		.\] 
		\[
		z^2-7z+2iz-7z-2iz+7-14i+14i+4=0
		.\] 
		 \[
		z^2-14z+53=0
		.\] 

		\textbf{Quick (preferred) way:}
		\[
			(z-(7-2i))(z-(7+2i))=(z-7-2i)(z-7+2i)
		.\] 
		This is a difference of two squares:
		\[
			=(z-7)^2-(2i)^2
		.\] 
		\[
			z^2-14z+47-(-4)
		.\] 
		\[
		=z^2-14z+53
		.\] 

		So the equation is $z^2-14z+53=0$.
	\end{solution}
\end{frame}

\begin{frame}{Past Paper Question}
	\begin{problem}
		Given that $2-4i$ is a root of the equation
		 \[
		z^2+pz+q=0
		,\]
		where $p$ and $q$ are real constants,
		\begin{itemize}
			\item write down the other root of the equation,
			\item find the value of $p$ and the value of $q$.
		\end{itemize}
	\end{problem}
	
	\begin{solution}<2->
		$z_2=2+4i$
	\end{solution}
	\begin{solution}<3->
		$(z-2+4i)(z-2-4i)=0$

		$\implies z^2-4z+4-(-16)$

		$\implies z^2-4z+20=0$
	\end{solution}
\end{frame}

\begin{frame}{Roots of Cubics}
	\begin{definition}
		Any cubic (with coefficients of $x^3$ of 1) can be expressed as:
		\[
			y=(x-\alpha)(x-\beta)(x-\gamma)
		\]

		where $\alpha,\beta$ and  $\gamma$ are the roots.

	\end{definition}
	However, this 3 roots may not necessarily all be real...

	\noindent
	\begin{minipage}{.3\linewidth}
		\centering
		\adjustbox{max width=\linewidth}{
		\begin{tikzpicture}
					\begin{axis}[mlineplot,width=4cm,height=4cm,
						xmin= -4, xmax= 4,
						ymin= -4, ymax = 4,
						axis lines = middle,
					]
					\addplot[color=aa,domain=-4:4, samples=100] {(x+3)*(x-1)*(x-3)};
					\node[auto] at (axis cs:-3,0) {$\alpha$};
					\node[auto] at (axis cs:1,0) {$\beta$};
					\node[auto] at (axis cs:3,0) { $\gamma$};
					\end{axis}
			\end{tikzpicture}}
			\sol{All 3 roots are real.}
	\end{minipage}%
	\begin{minipage}{.3\linewidth}

		\centering
		\adjustbox{max width=\linewidth}{
		\begin{tikzpicture}
					\begin{axis}[mlineplot,width=4cm,height=4cm,
						xmin= -4, xmax= 4,
						ymin= -4, ymax = 4,
						axis lines = middle,
					]
					\addplot[color=aa,domain=-4:4, samples=100] {(x+3)*(x-1)*(x-3)+4};
					\node[auto] at (axis cs:-3,0) {$\alpha$};
					\end{axis}
			\end{tikzpicture}}
			\sol{\parbox{\linewidth}{1 real root, 2 complex roots.}}
	\end{minipage}%
	\begin{minipage}{.3\linewidth}

		\centering
		\adjustbox{max width=\linewidth}{
		\begin{tikzpicture}
					\begin{axis}[mlineplot,width=4cm,height=4cm,
						xmin= -4, xmax= 4,
						ymin= -4, ymax = 4,
						axis lines = middle,
					]
					\addplot[color=aa,domain=-4:4, samples=100] {(x+3)*(x-2)*(x-2)};
					\node[auto] at (axis cs:-3,0) {$\alpha$};
					\node[auto] at (axis cs:2,0) {$\beta=\gamma$};
					\end{axis}
			\end{tikzpicture}}
			\sol{3 real roots (one repeated)}
	\end{minipage}

Are there any other possibilities?

\begin{solution}<3->
	No: since cubics have a range of $-\infty$ to  $\infty$, it must cross the  $x$ axis. And it can't cross an even number of times, otherwise the cubic would start and end in the same vertical direction.
\end{solution}

\end{frame}

\begin{frame}{Past Paper Questions}
	\begin{problem}
		$x=2$ is one of the roots of the polynomial $x^3-x-6$. Find the other two roots.


	\end{problem}
	\begin{solution}<2->
		The factor theorem tells us that $(x-2)$ is a factor. We can then work out the factorisation is $(x-2)(x^2+2x+3)$. The other roots arise when the second bracket is zero. Using the quadratic formula, we obtain the roots: 
		\[
		x=-1 \pm i
		.\] 
	\end{solution}
	\begin{problem}
		$x=-\frac{1}{2}$ is one of the roots of the polynomial $2x^3-5x^2+5x+4$. Find the other two roots.
	\end{problem}
	\begin{solution}<3->
		By the factor theorem, $2x+1$ must be a factor. We can then work out that the factorisation is $(2x+1)(x^2-3x+4)$. The other roots occur when the second bracket is equal to zero. Using the quadratic formula, we obtain the roots:
		\[
			x=\frac{3\pm i}{2}.
		\] 
	\end{solution}
	\alert{You should know from the Algebra topic from you're A-level how to factorise a cubic once you've got one root by using your favourite method of polynomial division.}
\end{frame}

\begin{frame}{Finding Other Roots}
	\begin{definition}
		Remember that complex roots of polynomials always come in complex conjugate pairs.
	\end{definition}

	\begin{problem}
		$-1+2i$ is one of the roots of the cubic $ z^3-z^2-z-15$.

		Find the other two roots.
	\end{problem}
	\begin{solution}<2->
		The other complex root is the complex conjugate: $-1-2i$.

		Now expand  $(z-(-1+2i))(z-(-1-2i))=z^2+2z+5$.

		We can now work out that the factorisation is $(z^2+2z+5)(z-3)$ so the other root is 3.
	\end{solution}

	\begin{problem}
		$2-i$ is one of the roots of the cubic $z^3-11z+20$. Find the other two roots.
	\end{problem}
	\begin{solution}<3->
		$2+i$ is the complex conjugate and hence is the other complex root.

		$(z-(2+i))(z-(2-i))=z^2-4z+5$

		The factorisation is $(z^2-4z+5)(z+4)$, so the real root is $-4$.
	\end{solution}
\end{frame}

\begin{frame}{Test Your Understanding}
	\begin{problem}
		Given that $3+i$ is a root of the quartic equation,
		\[
		2z^{4} -2z^3-39z^2+120z-50=0
		.\] 
		Solve the equation completely.
	\end{problem}
	\begin{solution}<2->
		Another root is $3-i$.
\begin{flalign*}
	\text{So...} && (z-(3+i))(z-(3-i)) &= && \\
		  && &= z^2-(3+3)z+3^2+1^2 && \\
		  && &= z^2-6z+10 && \\
.\end{flalign*}
is a factor of $2z^{4} -2z^3-39z^2+120z-50$
\[
	(z^2-6z+10)(...)=2z^{4} -2z^3-39z^2+120z-50
.\] 
Use (mostly) common sense to determine the other bracket: \begin{itemize}
	\item It must start with $2z^2$ in order to get the $2z^4$ term in the expansion.
	\item It must end with  $-5$ to get the  $-50$ term.
	\item So we know the second bracket is of the form  $(2z^2+az-5).$
\end{itemize}

To work out the $a$ we need to compare either  $z^3,z^2$ or $z$ terms in the expansion, say for example the  $-3z^3$ term:
\[
a-12=-3 \implies a=9
.\] 
\[
	\therefore (z^2-6z+10)(2z^2+9z-5)=0
.\] 
Solving $2z^2+9z-5 \implies z=\frac{1}{2}, z=-5$.

So roots are $\frac{1}{2},-5,3+i,3-i$.
	\end{solution}
	\alert{You could also use algebraic long division...}
\end{frame}

\begin{frame}{Past Paper Question}
	\begin{problem}
		Given that 2 and $5+2i$ are roots of the equation,
		 \[
		x^3-12x^2+cx+d=0, \quad c,d \in \R,
		\]

		\begin{itemize}
			\item write down the other complex root of the equation.
			\item Find the value of $c$ and the value of $d$.
		\end{itemize}
	\end{problem}
	\begin{solution}<2->
		$5-2i$ is a root.
	\end{solution}
	\begin{solution}<3->
		$(x-(5+2i))(x-(5-2i))=x^2-10x+29$

		$x^3-12x^2+cx+d=(x^2-10x+29)(x-2)$

		$c=49, \quad \quad \quad \quad \quad \quad d=-58$
	\end{solution}
\end{frame}

\begin{frame}{Quartics}
	\begin{definition}
		Any quartic (with coefficient of $x^4$ of 1) can be expressed as:
		\[
			y=(x-\alpha)(x-\beta)(x-\gamma)(x-\delta),
		\]
		where $\alpha,\beta,\gamma$ and  $\delta$ are the roots.
	\end{definition}
	\noindent
	\begin{minipage}{.3\linewidth}
		\centering
		\adjustbox{max width=\linewidth}{
		\begin{tikzpicture}
					\begin{axis}[mlineplot,width=4cm,height=4cm,
						xmin= -4, xmax= 4,
						ymin= -4, ymax = 4,
						axis lines = middle,
					]
					\addplot[color=aa,domain=-4:4, samples=100] {(x+3)*(x+2)*(x+1)*(x-2)};
					\node[auto] at (axis cs:-3,0) {$\alpha$};
					\node[auto] at (axis cs:-2,0) {$\beta$};
					\node[auto] at (axis cs:-1,0) { $\gamma$};
					\node[auto] at (axis cs:2,0) { $\delta$};
					\end{axis}
			\end{tikzpicture}}
			\sol{\parbox{\linewidth}{4 real roots (some \\ potentially repeated)}}
	\end{minipage}%
	\begin{minipage}{.3\linewidth}
		\centering
		\adjustbox{max width=\linewidth}{
		\begin{tikzpicture}
					\begin{axis}[mlineplot,width=4cm,height=4cm,
						xmin= -4, xmax= 4,
						ymin= -4, ymax = 4,
						axis lines = middle,
					]
					\addplot[color=aa,domain=-4:4, samples=100] {(x+3)*(x+2)*(x+1)*(x-2)+4};
					\node[auto] at (axis cs:-1,0) {$\alpha$};
					\node[auto] at (axis cs:2,0) { $\beta$};
					\end{axis}
			\end{tikzpicture}}
			\sol{\parbox{\linewidth}{2 real roots, pair of \\ complex conjugate roots.}}
	\end{minipage}%
	\begin{minipage}{.3\linewidth}
		\centering
		\adjustbox{max width=\linewidth}{
		\begin{tikzpicture}
					\begin{axis}[mlineplot,width=4cm,height=4cm,
						xmin= -4, xmax= 4,
						ymin= -4, ymax = 4,
						axis lines = middle,
					]
					\addplot[color=aa,domain=-4:4, samples=100] {(x+3)*(x+2)*(x-2)*(x-3)+7};
					\end{axis}
			\end{tikzpicture}}
			\sol{\parbox{\linewidth}{No real roots. Two pairs \\ of complex conjugate \\ roots.}}
	\end{minipage}
\end{frame}

\begin{frame}{Test Your Understanding}
	\begin{problem}
		Show that $z^2+4$ is a factor of $z^4-2z^3+21z^2-8z+68$.

		Hence solve the equation $z^4-2z^3+21z^2-8z+68=0$.
	\end{problem}

	\begin{solution}<2->
		$z^4-2z^3+21z^2-8z+68=(z^2+4)(z^2+az+17)$

		Comparing $z^3$ terms:
		\[
		-2=a
		.\] 
		\[
			\therefore z^4-2z^3+21^2-8z+68=(z^2+4)(z^2-2z+17)
		.\] 

		Solving $z^2+4=0$:
		\[
		z^2=-4
		.\] 
		\[
		z=\pm 2i
		.\] 

		Solving $z^2-2z+17=0:$
		\[
		z=1\pm 4i
		.\] 
	\end{solution}
\end{frame}



\begin{frame}{Cubics and Quartics Exercises}
	\begin{itemize}
		\item Given that $1+2i$ is one of the roots of a quadratic equation with real coefficients, find the equation.

			\sol{$x^2-2x+5=0$}

		\item Given that $a+4i$, where  $a$ is real, is one of the roots of a quadratic equation with real
coefficients, find the equation.

\sol{ $x^2-2ax+a^2+16$}

\item Show that $x=3$ is a root of the equation  $2x^3-4x^2-5x-3=0$.

	Hence solve the equation completely.

	\sol{Roots are $3,-\frac{1}{2}+\frac{1}{2}i,-\frac{1}{2}-\frac{1}{2}i$.}

\item Given that $-4+i$ is one of the roots of the equation  $x^3+4x^2-15x-68=0$, solve the equation completely.

	\sol{Roots are $4,-4+i$ and  $-4-i$.}

\item Given that  $2+3i$ is one of the roots of the equation $x^4+2x^3-x^2+38x+130=0$, solve the equation completely.

	\sol{Roots are $2+3i,2-3i,-3+i$ and $-3-i$ }

\item Find the four roots of the equation $x^4-16=0.$ Show these roots on an Argand Diargram.

	\sol{Roots are  $2,-2,2i,-2i$.}
	\end{itemize}
\end{frame}


\end{document}
