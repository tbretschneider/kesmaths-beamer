\documentclass[8pt]{beamer}

\usepackage[utf8]{inputenc}                                                     
 \usetheme[block=fill,progressbar=foot,background=light]{metropolis}                                                               
%  \usecolortheme{crane}                                                       
  %\useinnertheme{circles}                                                         
  \usepackage[english]{babel}                                                     
  \usepackage{csquotes}                                                           
  \usepackage[T1]{fontenc}                                                        
  \usepackage{booktabs}                                                       \usepackage{pgfgantt}
  \usepackage{pifont}
  \usepackage{adfbullets}
  \usepackage{enumitem}
  \usepackage{amsmath}   
  \usepackage{tikz}
  \usepackage{amssymb}
  \usepackage{amsfonts}
  \usepackage{mathrsfs}   
  \usepackage{graphicx}
  \usepackage{adjustbox}
  \usepackage{varioref}
  \usepackage{probsoln}
  \usepackage{attachfile2}
  \usepackage{pgfplots}
\pgfplotsset{compat=newest}
  \usepackage[style=authoryear,backref=true]{biblatex}
 \usepackage[]{hyperref} 
  \graphicspath{{Graphics/}}
  \usepackage{multirow,array}
  \addbibresource{../Everything.bib}
  \usepackage{colortbl}
  \definecolor{aa}{RGB}{255, 124, 0}
  \definecolor{cc}{RGB}{230, 230, 230}    
  %\setbeamercolor{palette tertiary}{fg=aa,bg=cc}
  %\setbeamercolor{structure}{fg=cc}
  %\setbeamercolor{alerted text}{fg=red}
  
  %Information to be included in the title page:
  
  \usebackgroundtemplate{%
  \tikz[overlay,remember picture]{\node[scale=80,opacity=0.03, at=(current page.south east)] {\adfbullet{9}};}}
  
  \author[]{T. Bretschneider}
  
  \date[\today]{\today}

\usepackage{comment}
\usepackage{varwidth}

\newcommand{\mat}[4]{\left(\begin{array}{cc} #1 & #2 \\ #3 & #4 \\ \end{array}\right)}
\newcommand{\Q}{\mathbb{Q}}
\newcommand{\R}{\mathbb{R}}
\newcommand{\Z}{\mathbb{Z}}
\newcommand{\sol}[2][+]{
	\tikz[baseline]{\node[color=aa,fill=cc,rectangle,draw,anchor=base] {  {\onslide<#1->{#2}}  };}
}

\usetikzlibrary{positioning}
\usetikzlibrary{tikzmark}
\usetikzlibrary{shadings}
\usetikzlibrary{through}


\def\height{0.8cm}
\def\width{1.2cm}

		\newcommand{\keynode}[6]{\node[minimum height=\height,minimum width=\width,draw,rectangle,color=aa,fill=cc] (#3) at (#1,#2) {};
	\node[rectangle,minimum height=\height/2,minimum width=\width,above,color=aa] at (#3) {#3};
	\node[draw,rectangle,minimum height=\height/2,minimum width=\width/3,below,color=aa,fill=cc,inner sep =0cm] at (#3) {\footnotesize#4};
	\node[draw,rectangle,minimum height=\height/2,minimum width=\width/3,below,xshift=\height/2,color=aa,fill=cc,inner sep=0cm] at (#3) {\footnotesize#5};
	\node[draw,rectangle,minimum height=\height/2,minimum width=\width/3,below,xshift=-\height/2,color=aa,fill=cc,inner sep=0cm] at (#3) {\footnotesize#6}; }

\newenvironment{gantt}[3]{\begin{ganttchart}[#1,bar height=.6,bar top shift=.2,title/.style=  {draw=none},y unit chart=0.6cm,y unit title = 0.6cm,include title in canvas=false,group/.append style={draw=black,dashed},bar/.append style={fill=aa},inline,hgrid=true,Float1/.style={bar/.append style={fill=none,dashed},bar height=.8,bar top shift=0.1}]{#2}{#3}}{\end{ganttchart}}

\newenvironment{nicetable}[1]{\setlength\arrayrulewidth{0.5mm}
			\arrayrulecolor{white}
			\begin{tabular}{#1}}{\end{tabular}}
		
\setlist[itemize,1]{label={\color{aa}\huge\adfbullet{9}}}
\setlist[itemize, 2]{label={\color{aa}\large\adfbullet{9}}}

\newcommand\reshist{}
\def\reshist(#1)#2(#3)#4(#5)%
{\draw (axis cs:#1) rectangle (axis cs:#3) node [midway] {#5};}





%Information to be included in the title page:
\title[Pure]{{\color{aa}\Huge\adfbullet{9}}AL FM Pure}
\subtitle{Differentiating and Integrating Inverse Trigonometric Functions}
%\title{Differentiating and Integrating Inverse Trigonometric Functions}
\author{Yavuz}
\date{\today}

\begin{document}
	
	\setlength{\abovedisplayskip}{0pt}
	\setlength{\belowdisplayskip}{0pt}
	\setlength{\abovedisplayshortskip}{0pt}
	\setlength{\belowdisplayshortskip}{0pt}
	
	\frame{\titlepage}
	
	
	\begin{frame}[shrink]{Differentiating Inverse Functions }
		\alert<1>{At AL you are also expected to know that, when $y$ isn't the subject, it can sometimes be easier to find $\frac{dx}{dy}$ and then \emph{'turn it upside down'}}
		
		\begin{definition}
			$\dfrac{d}{dx}\left( \sin ^{-1}x\right) =\dfrac{1}{\sqrt{1-x^{2}}}$
		\end{definition}
		
		Let $y = \sin ^{-1}x$ so we are trying to find \sol{$\dfrac{dy}{dx}$}
		
		Then $\sin y = x$ or reversing the equation \sol{$x= \sin y$}
		
		Differentiating both sides with respect to $y$: \sol{$\dfrac{dx}{dy} = \cos y$}
		
		Because there are no $y$s in the question, there should be no $y$s in our answer: 
		
		$\cos y=$\sol{$\sqrt{1-\sin ^{2}y}=\sqrt{1-x^{2}}$} (because on line $2$ we had $x = \sin y$)
		
		$\therefore\dfrac{dy}{dx}=$\sol{$\dfrac{1}{\sqrt{1-x^{2}}}$}
	\end{frame}
	
	
	\begin{frame}[shrink]{Test Your Understanding}
		\begin{problem}
			Prove $\dfrac{d}{dx}\left( \cos ^{-1}x\right) =-\dfrac{1}{\sqrt{1-x^{2}}}$
		\end{problem}
		
		\begin{solution}<2->
			Let $y = \cos^{-1}x$ so we are trying to find $\dfrac{dy}{dx}$ 
			
			Then $\cos y = x$ or reversing the equation $x = \cos y$
			
			Differentiating both sides with respect to $y$: $\dfrac{dx}{dy} = -\sin y$
			
			$\therefore \dfrac{dy}{dx}=\dfrac{-1}{\sin y}$
			
			$\sin y=\sqrt{1-\cos y}=\sqrt{1-x^{2}}$ (because on line $2$ we had $x = \cos y$)
			
			$\therefore \dfrac{dy}{dx}=\dfrac{-1}{\sqrt{1-x^{2}}}$
			
		\end{solution}
	\end{frame}
	
	\begin{frame}[shrink]{Test Your Understanding}
		\begin{problem}
			Prove $\dfrac{d}{dx}\left( \tan ^{-1}x\right) =\dfrac{1}{1+x^{2}}$
		\end{problem}
		
		\begin{solution}<2->
			
			Let $y = \tan^{-1}x$ so we are trying to find $\dfrac{dy}{dx}$ 
			
			Then $\tan y = x$ or reversing the equation $x = \tan y$
			
			Differentiating both sides with respect to $y$: $\dfrac{dx}{dy} = \sec ^{2}y$
			
			$\therefore \dfrac{dy}{dx}=\dfrac{1}{\sec ^{2}\tikzmarknode{A}{y}}$ 
			
			\begin{tikzpicture}[overlay,remember picture]
				\draw[color=aa,<-] (A.east) --++ (2,0) node[align=center,right] {\alert{\small \parbox{.5\linewidth}{We could write this as $\cos^{2}y$ but as we   have a trig identity which relate $\tan$ and  $\sec$, it is simpler to leave it as it is.}}};
			\end{tikzpicture}

			
			$\sec ^{2}y=1+\tan ^{2}y = 1+x^{2}$ 
			
			$\therefore \dfrac{dy}{dx}=\dfrac{1}{1+x^{2}}$
		\end{solution}
\end{frame}
	
\begin{frame}{Test Your Understanding}
		\begin{problem}
			Given that $y = \text{arcsec} 2x$, show that $\dfrac{dy}{dx}=\dfrac{1}{x\sqrt{4x^{2}-1}}$
		\end{problem}
		
		\begin{solution}<2->
			$\sec y=2x$
			
			$x=\dfrac{\sec y}{2}$
			
			$\dfrac{dx}{dy}=\dfrac{\sec y\tan y}{2}$
			
			$\dfrac{dy}{dx}=\dfrac{2}{\sec y\tan y}=\dfrac{2}{\sec \sqrt{\sec ^{2}y-1}}=\dfrac{2}{2x \sqrt{4x^{2}-1}}$
			
			$=\dfrac{1}{x\sqrt{4x^{2}-1}}$
		\end{solution}
\end{frame}
	
	
	
\begin{frame}[shrink]{Test Your Understanding}
		\begin{problem}
			Given that $y = arctan \left( \dfrac{1-x}{1+x}\right)$, find $\dfrac{dy}{dx}$.
		\end{problem}
		
		\begin{solution}<2->
			Let $u = \dfrac{1-x}{1+x}$, then using the quotient rule, $\dfrac{du}{dx}=\dfrac{\left( 1-x\right) \left( -1\right) -\left( 1-x\right) \left( 1\right) }{\left( 1+x\right) ^{2}}=-\dfrac{2}{\left( 1+x\right) ^{2}}$
			\begin{flalign*}
			\text{Therefore, by the chain rule,} && \dfrac{dy}{dx} &=   \dfrac{1}{1+\left( \dfrac{1-x}{1+x}\right)^{2}} \times - \dfrac{2}{\left(1+x\right)^{2}} && \\	
		&& &= -\dfrac{2}{\left( 1+x\right) ^{2}+\left( 1-x\right) ^{2}} && \\
		&& &= -\dfrac{2}{2+2x^{2}} && \\
		&& &= -\dfrac{1}{1+x^{2}} && 
			\end{flalign*}
		\end{solution}
\end{frame}
	
	\begin{frame}[shrink]{Integrating Inverse Functions}
		\begin{definition}
			$\int \dfrac{1}{\sqrt{a^{2}-x^{2}}}dx=\sin ^{-1}\left( \dfrac{x}{a}\right) +C, \quad \left| x\right| <a$
		\end{definition}
		
		The restriction $\left| x\right| <a$ is to ensure that there is \sol{no negative square root}
		
		\alert{Think about what a sensible substitution might be so that you end up with something squared under the square root. Also looking at the answer gives a hint about the substitution.}
		
		\begin{solution}<2->
			Let $x=a\sin u$ then $\dfrac{dx}{du} = a\cos u$ so $dx = a\cos u\ du$.
			\begin{flalign*} 
			&&	\int \dfrac{1}{\sqrt{a^{2}-x^{2}}}\ dx &= \int \dfrac{1}{\sqrt{a^{2}-a^{2}\sin^{2}u}} a\cos u\ du && \\
			&& &= \int \dfrac{1}{a\sqrt{1-sin^{2}}u}a\cos u\ du  && \\
			&& &= \int \dfrac{1}{a\cos u}a\cos u\ d u && \\
			&& &= \int 1 \ du &&  \\
		&&	&= u+C && 
			\end{flalign*}
			But rearranging $x=a\sin u$ gives $\frac{x}{a} = \sin u$ and $u = \sin ^{-1}(\frac{x}{a})$
			
			Therefore $\int \dfrac{1}{\sqrt{a^{2}-x^{2}}}\ dx=\sin^{-1}\left( \dfrac{x}{a}\right) +C$
			
		\end{solution}
	\end{frame}
	
	\begin{frame}[shrink]{Integrating Inverse Functions}
		\begin{definition}
			$ \int \dfrac{1}{a^{2}+x^{2}}\ dx=\dfrac{1}{a}\tan ^{-1}\left( \dfrac{x}{a}\right) +C$
		\end{definition}
		\alert{Again look at the answer to think about what might be a good substitution.}
		
		\begin{solution}<2->
			Let $x=a\tan u$ then $\dfrac{dx}{du} = a \sec^{2} u$ so $dx = a \sec^{2} u \ du$
			
			\begin{align*}
			\int \dfrac{1}{1+x^{2}}\ dx &= \int \dfrac{1}{a^{2}+a^{2}\tan^{2}u}a \sec^{2}u \ du \\
			&= \int \dfrac{1}{a^{2}(1+\tan ^{2}u)}a \sec^{2} \ du \\
			&= \int \dfrac{1}{a^{2} \sec^{2} u}a \sec^{2}\ du\\
			&= \frac{1}{a}\int 1 \ du \\
			&= \frac{1}{a}u+C \\
			\end{align*}
			But rearranging $x=a\tan u$ gives $\frac{x}{a} = \tan u$ and $u = \tan ^{-1}(\frac{x}{a})$.
			
			Therefore $\int \dfrac{1}{a^{2}+x^{2}}\ dx=\dfrac{1}{a}\tan ^{-1}\left( \dfrac{x}{a}\right) +C$.
			
		\end{solution}
	\end{frame}
	
	\begin{frame}[shrink]{Test Your Understanding}
		\begin{problem}
			Find $\int \dfrac{4}{5+x^{2}}\ dx$
		\end{problem}
		\begin{solution}<2->
			\[ \int \frac{4}{5+x^{2}}\ dx = 4 \int \frac{1}{5+x^{2}}\ dx=\frac{4}{\sqrt{5}}\arctan \left( \frac{x}{\sqrt{5}}\right) +C \]
		\end{solution}
		
		\begin{problem}
			Find $\int \frac{1}{25+9x^{2}}\ dx$
		\end{problem}
		\alert{Hint: Factorise out $\frac{1}{9}$ so that you get left with a single $x^{2}$ in the integral}
		\begin{solution}<3->
			\[\int \frac{1}{25+9x^{2}}\ dx=\frac{1}{9}\int \frac{1}{\frac{25}{9}+x^{2}}\ dx=\frac{1}{9}\times \frac{1}{\frac{5}{1}}\arctan \left( \frac{\frac{x}{5}}{3}\right) +C=\frac{1}{15}\arctan \left( \frac{3x}{5}\right) +C\]
		\end{solution}
		
		\begin{problem}
			Showing full working, find $\int ^{\dfrac{\sqrt{3}}{4}}_{-\dfrac{\sqrt{3}}{4}}\dfrac{1}{\sqrt{3-4x^{2}}}\ dx$
		\end{problem}
		\begin{solution}<4->
			\[\int ^{\sqrt{3}/4}_{-\sqrt{3}/4}\dfrac{1}{\sqrt{3-4x^{2}}}\ dx=\int ^{\sqrt{3}/{4}}_{-\sqrt{3}/{4}} \dfrac{1}{\sqrt{4}\sqrt{\dfrac{3}{4}-x^{2}}}\ dx=\dfrac{1}{2}\left[ \arcsin \left( \dfrac{2x}{\sqrt{3}}\right) \right] _{-\frac{\sqrt{3}}{4}}^{\frac{\sqrt{3}}{4}}=\frac{\pi }{12}-\left( -\frac{\pi }{12}\right) =\frac{\pi }{6}\]
		\end{solution}
	\end{frame}
	
	\begin{frame}[shrink]{Test Your Understanding}
		
		\begin{problem}
			Find $\int \dfrac{x+4}{\sqrt{1-4x^{2}}}\ dx$
		\end{problem}
		\alert{Hint: Split up the numerator}
		\begin{solution}<2->
			\[\int \frac{x+4}{\sqrt{1-4x^{2}}}dx = \int \frac{x}{\sqrt{1-4x^{2}}}\ dx+4\int \frac{1}{\sqrt{1-4x^{2}}}\ dx \]
			
			Deal with each in turn:
			\[ \int \frac{x}{\sqrt{1-4x^{2}}}\ dx = \int x\left( 1-4x^{2}\right) ^{-\frac{1}{2}}\ dx=\frac{1}{4}\sqrt{1-4x^{2}}+C \]
			
			\[ 4\int \frac{1}{\sqrt{1-4x^{2}}}\ dx = 4\int \frac{1}{\sqrt{ 4( \frac{1}{4}-x^{2}) }}\ dx=2\int \frac{1}{\sqrt{\frac{1}{4}-x^{2}}}\ dx=2\arcsin 2x+C \]
			
			\[ \therefore \int \frac{x+4}{\sqrt{1-4x^{2}}}\ dx = \frac{1}{4}\sqrt{1-4x^{2}}+2\arcsin 2x+C\]
		\end{solution}
		
	\end{frame}
	
	\begin{frame}[shrink]{Quadratic Denominator that can't be Factorised}
		If you have a rational function where the denominator is a quadratic that can be factorised, you should factorise and use \sol{partial fractions} as in A Level.
		
		\begin{definition}
			When the denominator is a quadratic that can't be factorised, you should complete the square and use $\int \frac{1}{a^{2}+x^{2}} \ dx= \frac{1}{a} \arctan (\frac{x}{a}) + C$ formula.
		\end{definition}
		
		\begin{problem}
			Find the exact value of $\int ^{\sqrt{3}-3}_{-2}\dfrac{1}{x^{2}+6x+12}\ dx$
		\end{problem}
		
		\begin{solution}<2->
			\hspace*{-3cm}%
			\begin{align*}
			\int ^{\sqrt{3}-3}_{-2}\dfrac{1}{x^{2}+6x+12}dx &= \int ^{\sqrt{3}-3}_{-2}\dfrac{1}{\left( x+3\right) ^{2}+3}\ dx \hspace{10cm} \\
			& = \int ^{\sqrt{3}-3}_{-2}\dfrac{1}{\left( \sqrt{3}\right) ^{2}+\left( x+3\right) ^{2}}\ dx \\
			&= \dfrac{1}{\sqrt{3}}\left[ \arctan \left( \dfrac{x+3}{\sqrt{3}}\right) \right] _{-2}^{\sqrt{3}-\tikzmarknode{A}{3}} \\
			&= \dfrac{1}{\sqrt{3}}\left( \arctan 1-\arctan \left( -\dfrac{1}{\sqrt{3}}\right) \right) \\
			&= \dfrac{1}{\sqrt{3}}\left( \dfrac{\pi }{4}-\dfrac{\pi }{6}\right) \\
			&= \dfrac{\pi \sqrt{3}}{36} 
		\end{align*}
		
					\begin{tikzpicture}[overlay,remember picture]
			\draw[color=aa,<-] (A) --++ (1,0.5) node[right] {\parbox{.4\linewidth}{\alert{\small{If you can’t see this step in one go, you could substitute $u=x+3$. However when integrating, for a linear function of $x$ you can do what you would do for $x$ but divide by the derivative of that linear function.
							
							Here the linear function is $x+3$ so the derivative is $1$ so you don’t need to divide.
			}}}};
			
		\end{tikzpicture}
	\end{solution}
	\end{frame}
	
	\begin{frame}[shrink]{Partial Fractions}
		We can now integrate with partial fractions where one of the factors of the denominator is a quadratic that cannot be factorised.
		
		\alert{Recall from A Level that when you write as partial fractions, you must ensure you have the most general possible non-top heavy fraction, i.e. the ‘order’ (i.e. maximum power) of the numerator is one less than the denominator.}
		
		i.e. You should initially write $\dfrac{1}{x\left( x^{2}+1\right) }\dfrac{}{}\equiv \dfrac{A}{x}+\dfrac{Bx+C}{x^{2}+1}$
		
		\begin{problem}
			Show that $\int \dfrac{1+x}{x^{3}+9x}\ dx=A\ln \left( \dfrac{x^{2}}{x^{2}+1}\right) +B\arctan \left( \dfrac{x}{3}\right) +C$, where $A$ and $B$ are constants.
		\end{problem}
		\begin{solution}<2->
			\[ \int \dfrac{1+x}{x^{3}+9x}\ dx = \int \dfrac{1+x}{x(x^{2}+9)}\ dx = \dfrac{A}{x}+\dfrac{Bx+C}{x^{2}+9} \]
			
			$1+x\equiv \overline{}A\left( x^{2}+9\right) +x\left( Bx+C\right)$
			
			$x=0\Rightarrow A=\dfrac{1}{9}$ and $x=3\Rightarrow 4=2+9B+3C$ and $x=-3 \Rightarrow -2 = 2 +9B - 3C$
			
			Solving the simultaneous equations gives $B = -\frac{1}{9}$ and $C = 1$
			\begin{align*}
			\int \frac{1+x}{x^{3}+9x}\ dx &= \frac{1}{9} \int \frac{1}{x}\ dx - \frac{1}{9}\int\frac{x}{x^{2}+9}\ dx + \int \frac{1}{x^{2}+9}\ dx \\
			&= \frac{1}{9}\ln x-\frac{1}{18}\ln \left( x^{2}+9\right) +\frac{1}{3}\arctan \left( \frac{x}{3}\right) +C \\	
			&= \frac{1}{18}\ln \left( \frac{x^{2}}{x^{2}+1}\right) +\frac{1}{3}\arctan \left( \frac{x}{3}\right) +C \\
			\end{align*}
		\end{solution}
		
	\end{frame}
	
	
	\begin{frame}[shrink]{Partial Fractions}
		\alert{As in A Level, if the numerator has degree at least that of the denominator then you will have to do long division first.}
		
		\begin{problem}
			\begin{itemize}
				\item Express $\dfrac{x^{4}+x}{x^{4}+5x^{2}+6}$ as partial fractions.
				\item Hence find $\int \dfrac{x^{4}+x}{x^{4}+5x^{2}+6}\ dx.$
			\end{itemize}
		\end{problem}
		
		\begin{solution}<2->
			Using algebraic long division: $x+4 = (x^{4}+5x^{2}+6)1 -5x^{2}+x-6$
			
			$\therefore \dfrac{x^{4}+x}{x^{4}+5x^{2}+6}=1+\dfrac{-5x^{2}+x-6}{x^{4}+5x^{2}+6}$
			
			Since $x^{4}+5x^{2}+6 = (x^{2}+2)(x^{2}+3)$:$\dfrac{-5x^{2}+x-6}{x^{4}+5x^{2}+6} = \dfrac{Ax+B}{x^{2}+2}+\dfrac{Cx+D}{x^{2}+3}$
			
			Then $-5x^{2}+x-6 = (Ax+B)(x^{2}+3)+(Cx+D)(x^{2}+2)$
			
			Use $x=0,1,2,3$ to get four simultaneous equations which you can then solve on your calculator to get $A=1, B=2, C=-1 and D=-9$
			
			$\therefore \dfrac{x^{4}+x}{x^{4}+5x^{2}+6} = 1+\dfrac{x+4}{x^{2}+2}-\dfrac{x+9}{x^{2}+3}$
		\end{solution}
		
		\begin{solution}<3->
			$\int 1+\dfrac{x+4}{x^{2}+2}-\dfrac{x+9}{x^{2}+3}\ dx= \int 1+\dfrac{x}{x^{2}+2}+\dfrac{4}{x^{2}+2}-\dfrac{x}{x^{2}+3}-\dfrac{9}{x^{2}+3}\ dx$
			
			$=\ldots =x+\dfrac{1}{2}\ln \left| \dfrac{x^{2}+2}{x^{2}+3}\right| +2\sqrt{2}\arctan \left( \dfrac{x}{\sqrt{2}}\right) -3\sqrt{3}\arctan \left( \dfrac{x}{\sqrt{3}}\right) +C $
		\end{solution}
	\end{frame}
	
\end{document}
