% !TeX spellcheck = en_B
% !TeX encoding = UTF-8
\documentclass[8pt]{beamer}

 \usepackage[utf8]{inputenc}                                                     
 \usetheme[block=fill,progressbar=foot,background=light]{metropolis}                                                               
%  \usecolortheme{crane}                                                       
  %\useinnertheme{circles}                                                         
  \usepackage[english]{babel}                                                     
  \usepackage{csquotes}                                                           
  \usepackage[T1]{fontenc}                                                        
  \usepackage{booktabs}                                                       \usepackage{pgfgantt}
  \usepackage{pifont}
  \usepackage{adfbullets}
  \usepackage{enumitem}
  \usepackage{amsmath}   
  \usepackage{tikz}
  \usepackage{amssymb}
  \usepackage{amsfonts}
  \usepackage{mathrsfs}   
  \usepackage{graphicx}
  \usepackage{adjustbox}
  \usepackage{varioref}
  \usepackage{probsoln}
  \usepackage{attachfile2}
  \usepackage{pgfplots}
\pgfplotsset{compat=newest}
  \usepackage[style=authoryear,backref=true]{biblatex}
 \usepackage[]{hyperref} 
  \graphicspath{{Graphics/}}
  \usepackage{multirow,array}
  \addbibresource{../Everything.bib}
  \usepackage{colortbl}
  \definecolor{aa}{RGB}{255, 124, 0}
  \definecolor{cc}{RGB}{230, 230, 230}    
  %\setbeamercolor{palette tertiary}{fg=aa,bg=cc}
  %\setbeamercolor{structure}{fg=cc}
  %\setbeamercolor{alerted text}{fg=red}
  
  %Information to be included in the title page:
  
  \usebackgroundtemplate{%
  \tikz[overlay,remember picture]{\node[scale=80,opacity=0.03, at=(current page.south east)] {\adfbullet{9}};}}
  
  \author[]{T. Bretschneider}
  
  \date[\today]{\today}

\usepackage{comment}
\usepackage{varwidth}

\newcommand{\mat}[4]{\left(\begin{array}{cc} #1 & #2 \\ #3 & #4 \\ \end{array}\right)}
\newcommand{\Q}{\mathbb{Q}}
\newcommand{\R}{\mathbb{R}}
\newcommand{\Z}{\mathbb{Z}}
\newcommand{\sol}[2][+]{
	\tikz[baseline]{\node[color=aa,fill=cc,rectangle,draw,anchor=base] {  {\onslide<#1->{#2}}  };}
}

\usetikzlibrary{positioning}
\usetikzlibrary{tikzmark}
\usetikzlibrary{shadings}
\usetikzlibrary{through}


\def\height{0.8cm}
\def\width{1.2cm}

		\newcommand{\keynode}[6]{\node[minimum height=\height,minimum width=\width,draw,rectangle,color=aa,fill=cc] (#3) at (#1,#2) {};
	\node[rectangle,minimum height=\height/2,minimum width=\width,above,color=aa] at (#3) {#3};
	\node[draw,rectangle,minimum height=\height/2,minimum width=\width/3,below,color=aa,fill=cc,inner sep =0cm] at (#3) {\footnotesize#4};
	\node[draw,rectangle,minimum height=\height/2,minimum width=\width/3,below,xshift=\height/2,color=aa,fill=cc,inner sep=0cm] at (#3) {\footnotesize#5};
	\node[draw,rectangle,minimum height=\height/2,minimum width=\width/3,below,xshift=-\height/2,color=aa,fill=cc,inner sep=0cm] at (#3) {\footnotesize#6}; }

\newenvironment{gantt}[3]{\begin{ganttchart}[#1,bar height=.6,bar top shift=.2,title/.style=  {draw=none},y unit chart=0.6cm,y unit title = 0.6cm,include title in canvas=false,group/.append style={draw=black,dashed},bar/.append style={fill=aa},inline,hgrid=true,Float1/.style={bar/.append style={fill=none,dashed},bar height=.8,bar top shift=0.1}]{#2}{#3}}{\end{ganttchart}}

\newenvironment{nicetable}[1]{\setlength\arrayrulewidth{0.5mm}
			\arrayrulecolor{white}
			\begin{tabular}{#1}}{\end{tabular}}
		
\setlist[itemize,1]{label={\color{aa}\huge\adfbullet{9}}}
\setlist[itemize, 2]{label={\color{aa}\large\adfbullet{9}}}

\newcommand\reshist{}
\def\reshist(#1)#2(#3)#4(#5)%
{\draw (axis cs:#1) rectangle (axis cs:#3) node [midway] {#5};}







  \title[Pure]{{\color{aa}\Huge\adfbullet{9}}AL FM Pure}
  \subtitle{Limits, \textattachfile{Limits.tex}{(TeX)}}

\begin{document}

\setlength{\abovedisplayskip}{0pt}
\setlength{\belowdisplayskip}{0pt}
\setlength{\abovedisplayshortskip}{0pt}
\setlength{\belowdisplayshortskip}{0pt}


\frame{\titlepage}

\begin{frame}[shrink=5]{What are Limits?}
	A limit is the value a function gets closer to as the input gets closer to a given value

We talk about the limit as $x$ tends to $\infty$ (or $-\infty$) as being the value that the functions is
getting closer to as the $x$ gets larger and larger (or smaller and smaller).

\noindent
\begin{minipage}{.5\linewidth}
		\centering
		\adjustbox{max width=\linewidth}{
		\begin{tikzpicture}
					\begin{axis}[mlineplot,width=5cm,height=3cm,
						xmin= 0, xmax= 65,
						ymin= -5, ymax = 20,
						axis lines = middle,
					]
					\addplot[color=aa,domain=0:65, samples=100] {20*exp((-0.1*x))*cos(50*x)};
					\end{axis}
			\end{tikzpicture}}
			\[
				\lim_{x \to \infty }f(x)=\sol{0}
			.\] 
\end{minipage}%
\begin{minipage}{.5\linewidth}
	\centering                                                                                          
                  \adjustbox{max width=\linewidth}{                                                                   
                  \begin{tikzpicture}                                                                                 
                                          \begin{axis}[mlineplot,width=5cm,height=3cm,
						  xmin= -70, xmax= 0,
						  ymin= -5, ymax = 20,
						  axis lines = middle,
						  ]       
						  \addplot[color=aa,domain=-70:0, samples=200] {20*(1-30*exp(0.15*x)*(cos(50*x)^2+0.2)};
					  \end{axis}                                                                    
        \end{tikzpicture}}
		\[
			\lim_{x \to - \infty } f(x) = \sol{20}
		.\] 
	\end{minipage}

However, there may also be gaps in a function where it is undefined but the function is
getting closer to a certain value as it approaches that gap.

     \centering
	 \adjustbox{max width=\linewidth}{
	\begin{tikzpicture}
		\begin{axis}[mlineplot,width=7cm,height=3cm,
			xmin= -7, xmax= 7,
		    ymin= -0.5, ymax = 1.5,
			axis lines=middle,                                                                                                                              ]
				\addplot[color=aa,domain=-7:7, samples=100] {x/x};
		\end{axis}
	\end{tikzpicture}}

			    The function $f(x)=\frac{x}{x}$ is undefined at $x=0$ (because you can't compute $\frac{0}{0}$ but is 1 everywhere else. It should be clear to see that $\lim_{x \to 0} \frac{x}{x}=$\sol{1}.

\alert{You have seen such a limiting process in AL differentiation from first principles.}
\end{frame}

\begin{frame}[shrink=20]{Using Series Expansions to find a Limit}
	\begin{problem}
		Find $\lim_{x \to 0} \frac{x}{1-\sqrt{1-x} }$
	\end{problem}
	\begin{solution}<2->
		\noindent
		\begin{minipage}{.7\linewidth}
			Using the binomial expansion,
			\begin{align*}
				\sqrt{1-x} &= (1-x)^{\frac{1}{2}} \\
					   &= 1+\frac{1}{2}(-x)+\frac{\frac{1}{2}\times -\frac{1}{2}}{2!}(-x)^2+\frac{\frac{1}{2}\times -\frac{1}{2}\times -\frac{3}{2}}{3!}(-x)^3+\ldots \\
					   &= 1-\frac{1}{2}x-\frac{1}{8}x^2-\frac{1}{16}x^3+\ldots \\
				\text{Hence,} \quad \quad f(x)&=\frac{x}{1-(1-\frac{1}{2}x-\frac{1}{8}x^2-\frac{1}{16}x^3+\ldots}  \\
				&= \frac{x}{\frac{1}{2}x+\frac{1}{8}x^2+\frac{1}{16}x^3+\ldots} \\
				&= \frac{1}{\frac{1}{2}+\frac{1}{8}x+\frac{1}{16}x^2+\ldots} \\
			.\end{align*}
		
			It can now be seen that when $x\to 0$, all the terms in the denominator of the above expression, except the first, tend to zero. Hence,
			\[
				\lim_{x \to 0} f(x)=\frac{1}{\frac{1}{2}}=2
			.\] 
		\end{minipage}%
		\begin{minipage}{.2\linewidth}
			\alert{Notice that the key step in this question was that after we had used the binomial expansion then we cancelled an $x$ from the top and the bottom. Before this the value of the function at $x=0$ was $\frac{0}{0}$ but afterwards we got something that could be evaluated.}
		\end{minipage}
	\end{solution}

	\[
		(1+x)^n=1+nx+\frac{n(n-1)}{1.2}x^2+\ldots+\frac{n(n-1)\ldots(n-r+1)}{1.2\ldots r}x^r+\ldots \quad ( |x| < 1, n \in \Q)
	\] 
	
\end{frame}

\begin{frame}[shrink=20]{Using Series Expansions to find a Limit}
	\begin{problem}
		Find $\lim_{x \to 0} \frac{2\sin x -\sin 2x}{\cos x -\cos 2x}$.
	\end{problem}
	\begin{solution}<2->
		\begin{align*}
			\frac{2\sin x -\sin 2x}{\cos x -\cos 2x} &= \frac{2\left( x-\frac{x^3}{3!}+\frac{x^5}{5!}-\ldots \right)-\left( 2x - \frac{(2x)^3}{3!}+\frac{(2x)^5}{5!} -\ldots\right)  }{\left( 1-\frac{x^2}{2!}+\frac{x^4}{4!}-\ldots \right) -\left( 1- \frac{(2x)^2}{2!}+\frac{(2x)^4}{4!}-\ldots \right) } \\
								 &= \frac{\left( 2x-\frac{1}{3}x^3+\frac{1}{60}x^5 -\ldots\right) -\left( 2x-\frac{4}{3}x^3+\frac{4}{15}x^5-\ldots \right) }{\left( 1-\frac{x^2}{2}+\frac{x^4}{24}-\ldots \right) -\left( 1-2x^2+\frac{2}{3}x^4 -\ldots\right) } \\
								 &= \frac{x^3+\text{terms in $x^5$ and higher \tikzmarknode{A}{powers}}}{\frac{3}{2}x^2+\text{terms in $x^4$ and higher powers}} \\
								 &= \frac{x+\text{terms in $x^3$ and higher powers}}{\frac{3}{2}+\text{terms in $x^2$ and higher powers}}
		.\end{align*}

		As $x\to 0$, the numerator tends to zero and the denominator tends to $\frac{3}{2}$. Hence, 
		\[
		\lim_{x \to 0} \frac{2\sin x -\sin 2x}{\cos x -\cos 2x}=\frac{0}{\frac{3}{2}}=0
		.\] 

		\begin{tikzpicture}[overlay, remember picture]
			\draw[color=aa,<-] (A) --++ (2,0.5) node[right] {The mathematical notation for this is $O(x^5)$.};
		\end{tikzpicture}
	\end{solution}

\[
	\sin x = x- \frac{x^3}{3!}+\frac{x^5}{5!}-\ldots + (-1)^r \frac{x^{2r+1}}{(2r+1)!}+\ldots \quad \text{for all} 
	\quad x, \quad \cos x = 1- \frac{x^2}{2!}+\frac{x^4}{4!}-\ldots+(-1)^r \frac{x^{2r}}{(2r)!}+\ldots \quad \text{for all} \quad x.
\] 

\end{frame}

\begin{frame}[shrink=20]{Past Paper Question}
	\begin{problem}
		\begin{itemize}
			\item Find the first three non-zero in the expansion of $\frac{x}{\ln(1+x)}$ as a series in ascending powers of $x$.
			\item Hence find $\lim_{x \to 0} \left( \frac{1}{\ln (1+x)}-\frac{1}{x} \right) $.
		\end{itemize}

	\end{problem}

	\begin{columns}
	\begin{column}{.6\linewidth}
		\begin{solution}<2->
	\begin{align*}
		\frac{x}{\ln(1+x)} &= \frac{x}{x-\frac{x^2}{2}+\frac{x^3}{3}-\ldots} \\
		&= \frac{1}{1-\frac{x}{2}+\frac{x^2}{3}-\ldots} \\
		&= \left(1-\frac{x}{2}+\frac{x^2}{3}-\ldots \right)^{-1} \\
		&= 1- \left( -\frac{x}{2}+\frac{x^2}{3}-\ldots \right) + \frac{-\tikzmarknode{A}{1}\times -2}{2!}\left( -\frac{x}{2}+\frac{x^2}{3} \right)^2 +\ldots\\
		&= 1+\frac{x}{2}-\frac{x^2}{3}+\frac{x^2}{4}+\ldots \\
		&= 1+\frac{x}{2}-\frac{x^2}{12}+\ldots
	.\end{align*}

	\begin{tikzpicture}[overlay,remember picture]
		\draw[color=aa,<-] (A) --++ (0.8,1) node[above,align=center] {Using the binomial expansion \\ for $x^{-1}$ in the formula booklet \\ but with $\ln(1+x)$ instead of $x$}; 
		\end{tikzpicture}
			\end{solution}
	\end{column}
	\begin{column}{.4\linewidth}
		\begin{solution}<3->
	\begin{align*}
		\frac{1}{\ln (1+x)}-\frac{1}{x} &= \frac{1}{x}\left( \frac{x}{\ln(1+x)}-1 \right)  \\
						&= \frac{1}{x}\left( 1+\frac{x}{2}-\frac{1}{12}x^2+\ldots-1 \right)  \\
						&= \frac{1}{x}\left( \frac{x}{2}-\frac{1}{12}x^2+\ldots \right)  \\
						&= \frac{1}{2}-\frac{1}{12}x+\ldots \\
						&\to \text{as} \quad x \rightarrow 0
	.\end{align*}
\end{solution}
	\end{column}
	\end{columns}

	\[
		\ln(1+x)=x-\frac{x^2}{2}+\frac{x^3}{3}-\ldots+(-1)^{r+1} \frac{x^r}{r}+\ldots \quad (-1< x \leq 1)
	.\]
\end{frame}

\begin{frame}{L'H\^opital's Rule}
	L'H\^opital's Rule can be used to find limits for function of the form $\frac{f(x)}{g(x)}$ where $f(x)$ and $g(x)$ either both tend to zero or both tend to infinity.

	\begin{definition}
		\textbf{L'H\^opital's Rule} 

		If $f(c)=g(c)=0$ or  $\pm \infty$ then  $\lim_{x\to c} \frac{f(x)}{g(x)}=\lim_{x\rightarrow c} \frac{f'(x)}{g'(x)}$ \alert{Not in formula booklet}
	\end{definition}

	In other words if, when you plug in the limit, you get $\frac{0}{0}$ or $\frac{\infty}{\infty}$ then you can differentiate the numerator and the denominator and you will still get the same limit.

	\begin{definition}
		If after applying l'H\^opital's Rule we get something that is getting infinitely large as we get closer and closer to the limit, we say that the limit does not exist.
	\end{definition}

	\begin{exampleblock}{Find $\lim_{x\to 0} \frac{\sin x}{x^2}$}
		We can use l'H\^opital's rule because $f(0)=\sin 0 = 0$ and  $g(0)=0^2=0$

		$\frac{f'(x)}{g'(x)}=\frac{\cos x }{2x}$ 

		As $x\to 0$ then $\frac{\cos x}{2x}$ is getting larger and larger (tending to $\frac{1}{0}$ ) and so we say that the limit does not exist.
	\end{exampleblock}
\end{frame}

\begin{frame}[shrink=20]{L'H\^opital's Rule}
	\begin{problem}
		Evaluate $\lim_{x \to 1} \frac{5\ln x}{x-1}$.
	\end{problem}
	\alert{Here we can use L'H\^opital's rule because as $x \to 1$ both the numerator and denominator tend to 0.}
	\begin{solution}<2->
		Differentiating the numerator and denominator: $\lim_{x \to 1} \frac{5\ln x}{x-1}= \lim_{x \to 1} \frac{\frac{5}{x}}{1}=5 $.
	\end{solution}
	\begin{problem}
		Evaluate $\lim_{x \to \infty} \frac{4x}{e^x} $.
	\end{problem}
	\alert{Here we can use L'H\^opital's rule because as $x\to \infty$ both the numerator and denominator tend to $\infty$}
	\begin{solution}<3->
		Differentiating the numerator and denominator: $\lim_{x \to \infty} \frac{4x}{e^x}= \lim_{x \to \infty} = \frac{4}{e^x}=0  $.
	\end{solution}
	\begin{problem}
		Evaluate $\lim_{x \to \frac{\pi}{2}}(3\sec x - 3 \tan x) $.
	\end{problem}
	\alert{Sometimes you might have to manipulate the expression to make it in a form suitable for use in L'H\^opital's Rule}
	\begin{solution}<4->
		$3\sec x - 3 \tan x = \frac{3}{\cos x}- \frac{3 \sin x }{\cos x}=\frac{3-3\sin x}{\cos x}$.
		We can use L'H\^opital's Rule to evaluate $\lim_{x \to \frac{\pi}{2}} \frac{3-3\sin x}{\cos x}$ as numerator and denominator both tend to 0 as $x \to \frac{\pi}{2}$.

		Differentiating numerator and denominator: $\lim_{x \to \frac{\pi}{2}} \frac{3-3\sin x}{\cos x}= \lim_{x \to \frac{\pi}{2}} \frac{-3\cos x }{-\sin x}=\frac{0}{-1}=0$.
	\end{solution}
\end{frame}

\begin{frame}{L'H\^opital's Rule}
	Sometimes you might have to use L'H\^opital's Rule more than once.

	\begin{problem}
		Evaluate $\lim_{x \to 0} \left( \frac{1}{\ln(1+x)}-\frac{1}{x} \right) $.

	\end{problem}
	\begin{solution}<2->
		$\frac{1}{\ln(1+x)}-\frac{1}{x}=\frac{x-\ln(1+x)}{x\ln(1+x)}$.

		We can use L'H\^opital's Rule to evaluate $\lim_{x \to 0} \frac{x-\ln (1+x)}{ x\ln(1+x)}$ as numerator and denominator both tend to zero as $x \to 0$.

		Differentiating numerator and denominator:

		$\lim_{x \to 0} \frac{x-\ln(1+x)}{x\ln(1+x)}=\lim_{x \to 0} \frac{1-\frac{1}{1+x}}{\ln(1+x)+\frac{x}{1+x}}=\lim_{x \to 0} \frac{x}{(1+x)\ln(1+x)+x}$.

		We can use L'H\^opital's Rule again to evaluate $\lim_{x \to 0} \frac{x}{(1+x)\ln(1+x)+x}$ as numerator and denominator both tend to zero as $x \to 0$.

		Differentiating numerator and denominator:

		$\lim_{x \to 0} \frac{x}{(1+x)\ln(1+x)+x}=\lim_{x \to 0} \frac{1}{1+\ln(1+x)+1}=\frac{1}{2}$.
	\end{solution}
\end{frame}

\begin{frame}{Past Paper Question}
	\begin{problem}
		\begin{itemize}
			\item Find the value of $A$ for which we can use l'H\^opital's rule to evaluate the limit

				\[
				\lim_{x \to 2} \frac{x^2+Ax-2}{x-2}.
			\]
			\item For the value of $A$, give the value of the limit.
		\end{itemize}
	\end{problem}

	\begin{solution}<2->
		We need $\lim_{x \to 2} x^2+Ax-2=0$ so $4+2A-2=0$ so $A =-1$.
	\end{solution}
	\begin{solution}<3->
		Differentiating numerator and denominator: $\lim_{x \to 2} \frac{x^2-x-2}{x-2}=\lim_{x \to 2} \frac{2x-1}{1}=3$.
	\end{solution}
	
\end{frame}
\end{document}
